%%%%%%%%%%%%%%%%%%%%%%%%%%%%%%%%%%%%%%%%%%%%%%%%%%%%%%%%%%%%%%%%%%%%%%
% How to use writeLaTeX:
%
% You edit the source code here on the left, and the preview on the
% right shows you the result within a few seconds.
%
% Bookmark this page and share the URL with your co-authors. They can
% edit at the same time!
%
% You can upload figures, bibliographies, custom classes and
% styles using the files menu.
%
% Modifyied by Prof. MSc. Daniel Menin Tortelli
% Computer Science - URICER - Brazil
%%%%%%%%%%%%%%%%%%%%%%%%%%%%%%%%%%%%%%%%%%%%%%%%%%%%%%%%%%%%%%%%%%%%%%

\documentclass[12pt]{article}

\usepackage{article-config/sbc-template}

\usepackage{graphicx,url}

\usepackage[brazilian]{babel}
\usepackage[T1]{fontenc}
\usepackage[utf8]{inputenc}
\usepackage{multicol}
\usepackage{multirow}
\usepackage{graphicx,url}
\usepackage{amsmath,amssymb,amsfonts}
\usepackage{algorithmic}
\usepackage{textcomp}
\usepackage{url}
\usepackage{enumerate}
\usepackage{multirow}
\usepackage{soul}
\usepackage{verbatim}
\usepackage[table,xcdraw]{xcolor}
\usepackage[shortlabels]{enumitem}
\usepackage{scalefnt}
\usepackage{comment}
\usepackage{lscape}
\usepackage{fancyhdr}
\usepackage{article-config/pages-style}
\usepackage[super]{nth}
\usepackage[table]{xcolor}


\tolerance=1
\emergencystretch=\maxdimen
\hyphenpenalty=10000
\hbadness=10000

\sloppy


\curso{Ciência da Computação} % Preencha com o nome do curso
\local{URI Erechim/RS} % Local do evento
\ano{2025 } % Exemplo: 2025
\edicao{} % Exemplo: XXIII

\title{placeholder \\
\medskip
\small{
    \textit{Title: }
}
}

\author{Felipe Tomkiel Malacarne, Prof. Me. Marcos André Lucas aaaa}

\address{
    Universidade Regional Integrada do Alto Uruguai e das Missões \\
    Departamento de Engenharias e Ciência da Computação\\
  Caixa Postal 743 -- 99.709-910 -- Erechim -- RS -- Brasil
\email{101090@uricer.edu.br, mlucas@uricer.edu.br}
}


\begin{document}

\maketitle

\thispagestyle{plain}
\pagestyle{plain2}

\begin{abstract}
\end{abstract}

\keywords{Word 1, Word 2, Word 3, Word 4, Word 5.}

\begin{resumo}
\end{resumo}

\palavraschave{Palavra 1, Palavra 2, Palavra 3, Palavra 4, Palavra 5.}

\section{Introdução}


\subsection{Justificativa}


\subsection{Objetivos}

\subsubsection{Objetivo Geral}


\subsubsection{Objetivos Específicos}


\section{Referencial Teórico}
\label{sec:referencial}

\input{sections/metodology}
\section{Resultados e Discussão}
\label{sec:resultados}

Esta seção consolida as métricas obtidas nos testes de carga (k6) e no ensaio assíncrono com Cloud Tasks. As medições seguem os SLIs definidos na metodologia: latência P95, taxa de erro e tempo de drenagem do pipeline assíncrono.

\begin{table}[ht]
    \centering
    \caption{Resumo dos cenários de carga e conformidade com os SLOs}
    \label{tab:resultados_k6}
    \begin{tabular}{lccc}
        \toprule
        Cenário & Latência P95 & Throughput médio & Taxa de erro \\
        \midrule
        Leitura intensiva & 158~ms & 470 req/s & 0{,}00\% \\
        Mistura leitura/escrita & 4{,}03~s & 101 req/s & 0{,}00\% \\
        \bottomrule
    \end{tabular}
\end{table}

\subsection{Cenário de Leitura Intensiva}

Os 1.000 usuários virtuais do cenário de leitura mantiveram o backend em 470 requisições por segundo durante 17 minutos. A latência P95 permaneceu em 158~ms (Tabela \ref{tab:resultados_k6}), confortavelmente dentro do SLO de 250~ms, e nenhuma requisição falhou. Isso confirma que o caminho otimizado para consultas --- balanceador/CDN, API em Cloud Run, cache Redis e leituras do PostgreSQL --- absorve picos de leitura sem degradação perceptível. As séries de observabilidade mostraram CPU abaixo do limite de 10~vCPU imposto pela cota do Cloud Run, indicando que o gargalo potencial ainda estava distante.

\begin{figure}[ht]
    \centering
    \fbox{\parbox{0.85\linewidth}{\centering Placeholder para gráfico de latência/throughput do cenário de leitura.}}
    \caption{Latência e throughput observados no cenário de leitura intensiva.}
    \label{fig:latencia-leitura}
\end{figure}

\subsection{Cenário Misto (Leitura/Escrita)}

O cenário com 650 usuários alternando leituras e escritas expôs o comportamento do sistema na região de saturação: a latência P95 subiu para 4{,}03~s e excedeu o SLO, embora a taxa de erro tenha permanecido em 0{,}00\%. O throughput caiu para 101 req/s porque cada iteração acionava pipeline completo de escrita (validações, persistência em Cloud SQL, invalidação de cache e notificações em tempo real). As métricas dos bancos indicaram aumento no tempo de commit e mais contendas de bloqueio quando as transações tentavam atualizar simultaneamente os mesmos agregados, o que explica o crescimento das filas de requisições mesmo antes de atingir a cota máxima de vCPU. O teste confirma a necessidade de revisar o particionamento de dados (ex.: sharding lógico por usuário) ou aumentar o limite de instâncias da API para diluir o custo das operações de escrita.

\begin{figure}[ht]
    \centering
    \fbox{\parbox{0.85\linewidth}{\centering Placeholder para gráfico de latência e backlog do cenário misto.}}
    \caption{Evolução da latência e backlog no cenário leitura/escrita.}
    \label{fig:latencia-mix}
\end{figure}

\subsection{Processamento Assíncrono com Cloud Tasks}

O ensaio de filas injetou 51{,}58~mil tarefas em um único lote e monitorou a drenagem entre 01:08 e 01:18, totalizando aproximadamente 10~min. Isso corresponde a um ritmo médio de 86 tarefas/segundo por worker HTTP em Cloud Run, comprovando que a combinação Cloud Tasks + workers escala horizontalmente sem perder entregas. A latência ponta-a-ponta permaneceu estável porque o Redis operou apenas como backplane para eventos e não como fila, reduzindo a chance de contenção. Esse resultado valida que o pipeline assíncrono consegue absorver rajadas equivalentes ao dobro do tráfego diário da aplicação sem intervenção manual.

\begin{figure}[ht]
    \centering
    \fbox{\parbox{0.85\linewidth}{\centering Placeholder para gráfico de backlog e throughput das filas.}}
    \caption{Backlog do Cloud Tasks e taxa de consumo durante o ensaio de filas.}
    \label{fig:filas}
\end{figure}

\section{Conclusão}

\subsection{Principais Contribuições}

\subsection{Resultados Alcançados}

\subsection{Limitações}

\subsection{Trabalhos Futuros}


\bibliographystyle{article-config/sbc}
\bibliography{Referencias/referencias}

\end{document}
