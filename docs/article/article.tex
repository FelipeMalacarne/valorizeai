%%%%%%%%%%%%%%%%%%%%%%%%%%%%%%%%%%%%%%%%%%%%%%%%%%%%%%%%%%%%%%%%%%%%%%
% How to use writeLaTeX:
%
% You edit the source code here on the left, and the preview on the
% right shows you the result within a few seconds.
%
% Bookmark this page and share the URL with your co-authors. They can
% edit at the same time!
%
% You can upload figures, bibliographies, custom classes and
% styles using the files menu.
%
% Modifyied by Prof. MSc. Daniel Menin Tortelli
% Computer Science - URICER - Brazil
%%%%%%%%%%%%%%%%%%%%%%%%%%%%%%%%%%%%%%%%%%%%%%%%%%%%%%%%%%%%%%%%%%%%%%

\documentclass[12pt]{article}

\usepackage{article-config/sbc-template}

\usepackage{graphicx,url}

\usepackage[brazilian]{babel}
\usepackage[T1]{fontenc}
\usepackage[utf8]{inputenc}
\usepackage{multicol}
\usepackage{multirow}
\usepackage{graphicx,url}
\usepackage{amsmath,amssymb,amsfonts}
\usepackage{algorithmic}
\usepackage{textcomp}
\usepackage{url}
\usepackage{enumerate}
\usepackage{multirow}
\usepackage{soul}
\usepackage{verbatim}
\usepackage[table,xcdraw]{xcolor}
\usepackage[shortlabels]{enumitem}
\usepackage{scalefnt}
\usepackage{comment}
\usepackage{lscape}
\usepackage{fancyhdr}
\usepackage{article-config/pages-style}
\usepackage[super]{nth}
\usepackage[table]{xcolor}
\usepackage{longtable}
\usepackage{booktabs}
\usepackage{tikz}
\usetikzlibrary{positioning,calc}


\tolerance=1
\emergencystretch=\maxdimen
\hyphenpenalty=10000
\hbadness=10000

\sloppy


\curso{Ciência da Computação} % Preencha com o nome do curso
\local{URI Erechim/RS} % Local do evento
\ano{2025 } % Exemplo: 2025
\edicao{} % Exemplo: XXIII

\title{ValorizeAI: Documentação e Validação de uma Arquitetura Serverless Elasticamente Gerenciada \\
\medskip
\small{
    \textit{Title: ValorizeAI: Documenting and Validating a Managed Serverless Architecture}
}
}

\author{Felipe Tomkiel Malacarne, Prof. Me. Marcos André Lucas}

\address{
    Universidade Regional Integrada do Alto Uruguai e das Missões \\
    Departamento de Engenharias e Ciência da Computação\\
  Caixa Postal 743 -- 99.709-910 -- Erechim -- RS -- Brasil
\email{101090@uricer.edu.br, mlucas@uricer.edu.br}
}


\begin{document}

\maketitle

\thispagestyle{plain}
\pagestyle{plain2}

\begin{abstract}
    Digital financial workloads demand elastic architectures that remain observable under sustained stress. This paper documents and validates ValorizeAI, a serverless/containerized platform built on Cloud Run, Cloud SQL, Redis, Cloud Tasks, and WebSockets. Infrastructure-as-code modules provisioned replicable testbeds, while k6 executed two load scenarios (read-only and read/write) against SLOs of P95 $\leq$ 300~ms and error rate $<$ 0.5\%. The read scenario sustained 1{,}000 virtual users (mean 470 req/s, peak $\approx$970 req/s) with P95 = 158~ms and zero failures; the mixed scenario held 226 req/s at 650 VUs (violating the SLO at $\approx$539 VUs) with P95 = 658~ms and p99 = 2.67~s once the 10-instance Cloud Run quota was exhausted. The asynchronous pipeline processed 51.58k Cloud Tasks in 10 minutes (86 tasks/s) without loss or duplication. These results show that the architecture comfortably serves read-heavy traffic and that further horizontal scaling or write-path optimizations are required before supporting higher write concurrency.
\end{abstract}

\keywords{Serverless Computing, Cloud Run, Financial Platforms, Performance Testing, Observability.}

\begin{resumo}
    Plataformas financeiras modernas exigem arquiteturas elásticas capazes de lidar com picos de carga transacional sem abrir mão de consistência forte nem observabilidade. Este trabalho documenta e valida o ValorizeAI, aplicação real de categorização automática construída sobre Google Cloud Run, Cloud SQL, Redis (Memorystore), Cloud Tasks e WebSockets. A pesquisa adota abordagem aplicada inspirada em SRE: os SLOs (latência P95 $\leq$ 300~ms, erro $<$ 0,5\%, disponibilidade $\geq$ 99{,}5\%) são definidos antecipadamente, a infraestrutura é provisionada como código e os testes k6 cobrem cenários de leitura intensiva, mix leitura/escrita e o pipeline assíncrono. O cenário de leitura sustentou 1{,}000 VUs (média de 470 req/s, pico de $\approx$970 req/s) com P95 = 158~ms; o cenário misto manteve 226 req/s em 650 VUs e violou o SLO em $\approx$539 VUs, elevando o P95 para 658~ms e o p99 para 2,67~s ao saturar a cota de 10 instâncias do Cloud Run. O pipeline assíncrono processou 51,58 mil tarefas do Cloud Tasks em 10 minutos (86 tarefas/s) sem perda ou duplicação. Conclui-se que a arquitetura atende confortavelmente workloads intensivos em leitura e que novas instâncias HTTP ou otimizações no caminho de escrita são necessárias para acomodar escritas altamente concorrentes.
\end{resumo}

\palavraschave{Computação Serverless, Cloud Run, Plataformas Financeiras, Testes de Carga, Observabilidade.}

\section{Introdução}
\label{sec:introducao}

O ValorizeAI, objeto deste estudo, surge como um caso completo para investigar a relação entre elasticidade e observabilidade. Trata-se de uma aplicação web modular que integra ingestão de dados, processamento síncrono e assíncrono e comunicação em tempo real, utilizando uma \textit{stack} moderna baseada em Cloud Run, Cloud SQL, Redis (Memorystore), Cloud Tasks, WebSockets e serviços auxiliares.

Aplicações digitais modernas — incluindo plataformas de e-commerce, serviços financeiros e sistemas colaborativos — enfrentam o desafio de absorver cargas de trabalho voláteis sem comprometer a experiência do usuário \cite{google_elasticity_2024}. A resposta predominante tem sido a \textit{elasticidade na nuvem}, isto é, a capacidade de alocar e desalocar recursos automaticamente conforme a demanda varia, substituindo o provisionamento manual por escalonamento em tempo real para evitar desperdícios e manter a responsividade.

Essa elasticidade, implementada com microsserviços, contêineres e paradigmas \textit{serverless}, introduz complexidade que só pode ser controlada com observabilidade avançada \cite{newrelic_observability_2023}. Logs, métricas e \textit{traces} tornam-se insumos para SLIs que, por sua vez, validam e ajustam políticas automáticas de escalonamento. Elasticidade e observabilidade formam, assim, um ciclo de \textit{feedback} que opera em horizontes temporais inviáveis para processos manuais, reforçando a necessidade de arquiteturas e práticas integradas.

\subsection{Justificativa e Problema de Pesquisa}
\label{sec:justificativa}

Workloads transacionais que envolvem ingestão intensa de dados, estados compartilhados e interfaces colaborativas, como os simulados pelo ValorizeAI, impõem requisitos rigorosos: consistência forte, rastreabilidade para auditoria e respostas de baixa latência mesmo sob variações abruptas de tráfego. Para atender a esses requisitos, arquiteturas modernas combinam componentes especializados, como CDNs e balanceadores globais, filas assíncronas orientadas a eventos, cache distribuído e comunicação persistente via WebSockets \cite{barri_scalability_2025, confluent_eda_2024, yadav_redis_leaderboard_2019, fernando_websocket_2025}.

Embora existam estudos pontuais sobre esses componentes, a literatura apresenta lacunas quanto à validação integrada de arquiteturas híbridas (CaaS + filas + WebSockets) em cenários reprodutíveis de carga \cite{wjaets_serverless_ml_2022, abad_serverless_gap_2021}. Trabalhos existentes tendem a focar em comparações de ferramentas de IaC \cite{pessa_iac_2023} ou no desempenho de microsserviços isolados \cite{hebbar_reactive_2025}, raramente considerando o comportamento do sistema completo.

Este trabalho busca preencher essa lacuna ao documentar a arquitetura do ValorizeAI e validar seu comportamento sob estresse de carga frente a SLOs definidos. A questão central investigada é: \textit{Como uma arquitetura híbrida e elástica, composta por Cloud Run, Redis, Cloud SQL, Cloud Tasks e WebSockets dedicados, se comporta sob condições intensas de carga, e como esse comportamento pode ser validado de forma reprodutível?}

\subsection{Objetivo Geral}
\label{sec:objetivo_geral}

Demonstrar, por meio de documentação técnica e experimentos de desempenho, que a arquitetura do ValorizeAI, composta por CDN, contêineres escalados horizontalmente, processamento assíncrono em filas, servidor de WebSockets, armazenamento de artefatos em \textit{buckets} e cache distribuído em Redis, sustenta os SLOs definidos para um produto transacional completo, mantendo código, infraestrutura e observabilidade versionados em repositório.

\subsection{Objetivos Específicos}
\label{sec:objetivos_especificos}

\begin{enumerate}
    \item Mapear a arquitetura \textit{end-to-end}, destacando o papel do balanceador/CDN, instâncias de contêineres, servidor WebSockets, filas assíncronas, \textit{buckets} e Redis.
    \item Documentar o desenvolvimento dos módulos críticos do sistema, incluindo ingestão de dados, automações, notificações e painéis em tempo real.
    \item Planejar e executar testes de carga (k6, cenários de leitura e leitura/escrita) e testes assíncronos, validando a arquitetura frente aos SLOs definidos.
    \item Interpretar os resultados e propor otimizações relacionadas a desempenho, elasticidade e custo.
\end{enumerate}

\subsection{Contribuições Tangíveis}
\label{sec:contribuicoes}

\begin{enumerate}
    \item Arquitetura documentada e replicável.
    \item Infraestrutura reprodutível (Terraform, Docker, Makefile).
    \item Cenários de desempenho registrados e transparentes (k6).
    \item Validação do pipeline assíncrono baseado em Cloud Tasks.
\end{enumerate}

Em conjunto, essas contribuições formam um pacote completo de replicação, código, automações e experimentos, para avaliações futuras de workloads transacionais em ambientes CaaS.

\vspace{0.5em}
O restante deste artigo está organizado da seguinte forma:
a Seção~\ref{sec:relacionados} apresenta os trabalhos relacionados;
a Seção~\ref{sec:fundamentacao} discute a fundamentação teórica;
a Seção~\ref{sec:metodologia} descreve a metodologia experimental;
a Seção~\ref{sec:implementacao} detalha a implementação do ValorizeAI;
a Seção~\ref{sec:resultados} consolida os resultados e análises;
e, por fim, a Seção~\ref{sec:conclusao} apresenta as conclusões e trabalhos futuros.

\section{Trabalhos Relacionados}
\label{sec:relacionados}

A arquitetura investigada neste trabalho situa-se na interseção de três eixos centrais da literatura em sistemas distribuídos:
(i) paradigmas de execução em nuvem,
(ii) padrões arquiteturais para desempenho e resiliência e
(iii) metodologias de validação empírica.
Esta seção revisa o estado da arte nesses eixos para posicionar a contribuição do ValorizeAI e evidenciar a lacuna identificada na seção de Introdução.

\subsection{Paradigmas de Execução: Serverless (FaaS) e Contêineres Gerenciados (CaaS)}
\label{sec:paradigmas}

A escolha do paradigma de execução é um dos fatores determinantes para sistemas elásticos modernos. A literatura recente compara amplamente \textit{Functions-as-a-Service} (FaaS) e \textit{Containers-as-a-Service} (CaaS). O FaaS, amplamente representado por AWS Lambda e Google Cloud Functions, oferece escalonamento automático e faturamento por execução, mas apresenta limitações para aplicações \textit{stateful} ou de longa duração devido ao \textit{cold start}, ao isolamento elevado e à efemeridade das instâncias \cite{sonawane_serverless_review_2024, datadog_serverless_containers_2024}. Esses aspectos o tornam inadequado para componentes persistentes como servidores WebSocket.

Por outro lado, o CaaS — como Google Cloud Run ou AWS Fargate — mantém a elasticidade do FaaS, mas preserva o controle sobre o ambiente do contêiner e comporta processos contínuos \cite{lloyd_serverless_investigation_2018}. Essa característica é essencial para o servidor de WebSockets do ValorizeAI (Laravel Reverb), que requer conexões persistentes e compartilhamento de estado via Redis. Assim, a literatura sustenta a escolha do CaaS como paradigma mais adequado para arquiteturas híbridas que combinam serviços \textit{stateless} e componentes de longa duração.

\subsection{Padrões Arquiteturais para Desempenho e Resiliência}
\label{sec:padroes}

Para que a arquitetura do ValorizeAI atenda aos SLOs definidos,
ela precisa combinar padrões síncronos e assíncronos capazes de manter responsividade,
disponibilidade e escalabilidade diante de variações de carga.
A literatura descreve esses padrões como peças complementares, filas assíncronas absorvem rajadas,
enquanto canais em tempo real sustentam interações colaborativas, e fornece referências diretas para as escolhas arquiteturais adotadas neste trabalho.

\subsubsection{Arquiteturas Orientadas a Eventos (EDA) e Filas Assíncronas}
\label{sec:eda}

Arquiteturas orientadas a eventos promovem desacoplamento, resiliência e absorção de picos de carga ao delegar tarefas para filas assíncronas \cite{confluent_eda_2024}. Estudos comparativos analisam o comportamento de diferentes \textit{brokers} — como RabbitMQ, Apache Kafka e Pulsar — frente a cenários com variação de tamanho de mensagens e taxas de publicação \cite{thepphakan_pulsar_rabbitmq_2025}. Esses trabalhos demonstram que a escolha da fila deve refletir o perfil do \textit{workload}: baixa latência para eventos pequenos ou alto \textit{throughput} contínuo para fluxos intensivos.

A arquitetura do ValorizeAI segue essa abordagem ao integrar Cloud Tasks para desacoplar operações pesadas do ciclo de requisição HTTP, garantindo responsividade mesmo sob carga.

\subsubsection{Comunicação em Tempo Real e Cache Distribuído}
\label{sec:realtime}

Para requisitos de tempo real, a literatura destaca o uso combinado de WebSockets e cache distribuído. Redis é amplamente utilizado como memória compartilhada de baixa latência e como mecanismo Pub/Sub para orquestrar a entrega de eventos entre múltiplas instâncias \cite{yadav_redis_leaderboard_2019}. Em ambientes elásticos — como Cloud Run — o Redis atua como \textit{backplane} que mantém a consistência entre instâncias efêmeras ao distribuir mensagens para clientes conectados.

Estudos recentes analisam o impacto desse modelo em métricas de \textit{throughput} e latência RTT, validando suas vantagens para sistemas colaborativos e dashboards em tempo real \cite{fernando_websocket_2025, twine_websocket_scaling_2022}. Esse padrão é consistente com o design do ValorizeAI, cujo servidor WebSocket persistente publica e assina eventos via Redis para garantir consistência e escalabilidade horizontal.

\subsection{Metodologias de Validação Empírica}
\label{sec:metodologias_validacao}

Validar arquiteturas distribuídas exige metodologias reprodutíveis e alinhadas a objetivos de serviço.

\subsubsection{Infraestrutura como Código (IaC)}
\label{sec:iac}

IaC é amplamente adotado para garantir reprodutibilidade e eliminar variações de ambiente em experimentos de desempenho \cite{pessa_iac_2023}. Estudos analisam ferramentas como Terraform e AWS CDK em termos de eficiência, expressividade e redução de deriva de configuração \cite{guerriero_iac_2019}. Entretanto, tais estudos frequentemente focam na ferramenta, e não no sistema provisionado — o que contrasta com este trabalho, no qual IaC é utilizado como base para experimentos e revalidações sucessivas do ambiente completo (CaaS, filas, Redis e banco de dados).

\subsubsection{Validação por SLOs e Testes de Carga}
\label{sec:k6_slo}

A Engenharia de Confiabilidade de Sites (SRE) recomenda validação orientada a SLOs para medir sucesso operacional \cite{mccoy_slo_2020}. A literatura recente adota ferramentas modernas como o k6 para simular perfis realistas de carga e avaliar latência, erro e saturação \cite{cervone_k6_2024}. O trabalho de Hebbar \cite{hebbar_reactive_2025}, por exemplo, utiliza k6 para validar priorização de tráfego em APIs reativas, monitorando percentis de latência e comportamento sob estresse.

A estratégia utilizada pelo ValorizeAI — definição de SLOs, instrumentação do sistema e execução de cenários de carga reprodutíveis — encontra suporte direto nesses estudos, reforçando a adequação da metodologia adotada.

\subsection{Síntese e Lacuna de Pesquisa}
\label{sec:lacuna}

A literatura revisada é rica, mas fragmentada: há estudos sobre FaaS vs. CaaS \cite{lloyd_serverless_investigation_2018}, comparações de \textit{brokers} de EDA \cite{thepphakan_pulsar_rabbitmq_2025}, análises de ferramentas de IaC \cite{pessa_iac_2023} e validações de desempenho específicas usando k6 e SLOs \cite{hebbar_reactive_2025}. Contudo, conforme discutido por Abad et al. \cite{abad_serverless_gap_2021}, falta uma análise integrada de arquiteturas híbridas que combinem todos esses elementos em um único sistema reprodutível.

A contribuição do ValorizeAI está exatamente nessa integração: uma arquitetura CaaS orientada a eventos com WebSockets persistentes, cache Redis e filas assíncronas, provisionada integralmente via IaC e validada com metodologia alinhada ao estado da arte. Os estudos analisados servem como blocos isolados; este trabalho, por sua vez, propõe uma validação \textit{end-to-end} que abrange paradigma, padrões arquiteturais e metodologia experimental.

\section{Fundamentação Teórica}
\label{sec:fundamentacao}

Segue uma síntese do vocabulário e das bases conceituais utilizados no design, implementação e validação do ValorizeAI, cobrindo princípios de design de software, arquitetura dos componentes e fundamentos de engenharia de confiabilidade.

\subsection{Princípios de Design de Software}
\label{sec:principios_design}

O ValorizeAI adota uma abordagem de "arquitetura limpa", segregando responsabilidades com base em princípios estabelecidos de design de software.

\subsubsection{Clean Architecture}
\label{sec:clean_architecture}

Formalizada por Robert C. Martin, a \textit{Clean Architecture} (Arquitetura Limpa) é um modelo arquitetural que advoga pela separação de interesses \cite{martin_clean_2017}. Seu objetivo é criar sistemas que sejam: (1) Independentes de frameworks; (2) Testáveis; (3) Independentes da interface do usuário (UI); e (4) Independentes do banco de dados \cite{martin_clean_2017}.

O pilar central dessa arquitetura é a \textit{Regra da Dependência} (The Dependency Rule). Esta regra estipula que as dependências do código-fonte devem apontar exclusivamente "para dentro" --- de camadas de baixo nível (detalhes voláteis, como frameworks e bancos de dados) para camadas de alto nível (políticas de negócio estáveis e abstrações) \cite{martin_clean_2017}. No ValorizeAI, isso se manifesta na separação das regras de negócio (localizadas em \textit{Actions} ou \textit{Queries}) da lógica do framework (Controladores Laravel) ou da persistência (Modelos Eloquent).

\subsubsection{Domain-Driven Design (DDD)}
\label{sec:ddd}

O \textit{Domain-Driven Design} (DDD), introduzido por Eric Evans, é uma abordagem para o desenvolvimento de software que se concentra em modelar o software para corresponder a um domínio de negócio complexo \cite{evans_ddd_2003}. O DDD é essencial para gerenciar a complexidade em sistemas como o ValorizeAI. Os conceitos-chave utilizados neste trabalho incluem:

\begin{itemize}
    \item \textbf{Linguagem Ubíqua (Ubiquitous Language):} Um vocabulário compartilhado e rigoroso, desenvolvido em colaboração entre os desenvolvedores e os especialistas do domínio (usuários). Essa linguagem é usada em todas as comunicações e reflete-se diretamente no código (nomes de classes, métodos e variáveis) \cite{evans_ddd_2003}.
    \item \textbf{Contexto Delimitado (Bounded Context):} A fronteira explícita dentro da qual um modelo de domínio e sua Linguagem Ubíqua são aplicáveis e consistentes \cite{evans_ddd_2003}.
    \item \textbf{Agregado (Aggregate):} Um cluster de objetos de domínio (Entidades e Objetos de Valor) que é tratado como uma única unidade para fins de consistência de dados. Um Agregado possui uma raiz (a \textit{Aggregate Root}), que é o único ponto de entrada para modificações dentro do Agregado, garantindo que todas as regras de negócio (invariantes) sejam aplicadas \cite{evans_ddd_2003}.
\end{itemize}

\subsubsection{Padrões de Comunicação e Segregação}
\label{sec:cqrs_dto}

Para implementar a Regra da Dependência e gerenciar o fluxo de dados, o ValorizeAI utiliza padrões de segregação e transferência de dados.

\begin{itemize}
    \item \textbf{DTO (Data Transfer Object):} Conforme popularizado por Martin Fowler, um DTO é um objeto simples, sem comportamento, cujo único propósito é transferir dados entre subsistemas ou camadas \cite{fowler_peaa_2002}. Em arquiteturas distribuídas ou em camadas, os DTOs são usados para agregar múltiplas chamadas em uma única, reduzindo a latência da rede e desacoplando os modelos internos (domínio) dos modelos de visualização (API/UI).
    \item \textbf{CQRS (Command Query Responsibility Segregation):} Um padrão, descrito por Martin Fowler \cite{fowler_cqrs_2011} e Greg Young, que propõe a segregação dos modelos de dados e da lógica de aplicação em duas categorias: \textit{Commands} (operações que alteram o estado, ou seja, escritas) e \textit{Queries} (operações que leem o estado). O ValorizeAI adota esse princípio através da separação explícita de \textit{Actions} (Commands) e \textit{Queries} (Queries), permitindo otimizações distintas para os caminhos de escrita e leitura.
\end{itemize}

\subsection{Arquitetura e Componentes da Aplicação}
\label{sec:componentes_app}

A infraestrutura do ValorizeAI é composta por serviços gerenciados na nuvem, escolhidos por suas características de elasticidade e desempenho.

\subsubsection{Google Cloud Run e Cloud Tasks}
\label{sec:cloud_run}

O Google Cloud Run é uma plataforma de computação CaaS (Container-as-a-Service) totalmente gerenciada. Ele permite a execução de contêineres \textit{stateless} que escalam horizontalmente de forma automática, com a capacidade de escalar até zero instâncias quando não há tráfego, eliminando custos ociosos \cite{google_elasticity_2024}. O serviço foi escolhido por combinar a elasticidade típica de funções serverless com a flexibilidade dos contêineres, executando tanto os serviços web \textit{stateless} do ValorizeAI quanto o servidor \textit{stateful} de WebSockets.

O Google Cloud Tasks é o serviço de enfileiramento de tarefas gerenciado. Ele é usado para implementar o processamento assíncrono (EDA), permitindo que a aplicação principal (síncrona) enfileire tarefas de longa duração (ex: processamento de lotes) para execução em \textit{workers} separados, garantindo resiliência e baixa latência na resposta ao usuário.

\subsubsection{Laravel Reverb (WebSockets)}
\label{sec:reverb}

O Laravel Reverb é o servidor WebSocket oficial de primeira-parte para aplicações Laravel, projetado para comunicação em tempo real de alto desempenho \cite{laravel_reverb_docs_2025}. Ele utiliza o protocolo Pusher, integrando-se nativamente ao sistema de \textit{broadcasting} do Laravel para facilitar o envio de notificações \textit{push} aos clientes conectados.

A característica arquitetural mais importante do Reverb para este TCC é seu suporte à escalabilidade horizontal. Para operar em um ambiente elástico como o Cloud Run (com múltiplas instâncias de servidor), o Reverb utiliza um \textit{backplane} de mensagens, que no caso do ValorizeAI é implementado com o Redis (detalhado na subseção sobre cache) \cite{laravel_reverb_docs_2025}.

\subsubsection{Redis (Remote Dictionary Server)}
\label{sec:redis}

O Redis (Remote Dictionary Server) é um armazenamento de estrutura de dados em memória, de código aberto, usado como banco de dados, \textit{cache} e \textit{message broker} \cite{kleppmann_ddia_2017}. No contexto da arquitetura ValorizeAI, o Redis desempenha dois papéis críticos e distintos, ambos fundamentais para o desempenho do sistema:

\begin{enumerate}
    \item \textbf{Cache de Baixa Latência:} O Redis é usado como um \textit{cache} para dados frequentemente acessados (ex: painéis, dados de sessão). Sua operação em memória permite latências de leitura e escrita na ordem de submilissegundos, reduzindo drasticamente a carga sobre o banco de dados PostgreSQL e melhorando a responsividade das \textit{Queries} \cite{yadav_redis_leaderboard_2019}.
    \item \textbf{Backplane Pub/Sub:} O Redis fornece um mecanismo de Publicação/Subscrição (Pub/Sub) de alto desempenho. Este mecanismo é utilizado como o \textit{backplane} do Laravel Reverb. Quando uma instância do servidor (Instância A) precisa notificar um usuário que está conectado via WebSocket a outra instância (Instância B), a Instância A publica a mensagem em um canal Redis. Todas as outras instâncias, incluindo a Instância B, estão inscritas nesse canal, recebem a mensagem e a retransmitem aos seus clientes WebSocket conectados localmente.
\end{enumerate}

\subsection{Engenharia de Confiabilidade de Sites (SRE)}
\label{sec:sre}

A metodologia de validação deste trabalho é baseada nos princípios de Engenharia de Confiabilidade de Sites (SRE), popularizados pelo Google \cite{google_sre_book_main}. O SRE trata as operações de infraestrutura como um problema de engenharia de software, utilizando métricas rigorosas para equilibrar a inovação (velocidade de desenvolvimento) com a confiabilidade do serviço.

\subsubsection{SLIs, SLOs e Orçamentos de Erro}
\label{sec:slo_sli}

Os conceitos centrais do SRE utilizados para a validação do ValorizeAI são:

\begin{itemize}
    \item \textbf{SLI (Service Level Indicator):} Um indicador de nível de serviço é uma medida quantitativa de um aspecto da qualidade do serviço fornecido \cite{mccoy_slo_2020}. Os SLIs são métricas diretas do desempenho do sistema, como latência de requisição, taxa de erro ou \textit{throughput} do sistema \cite{google_sre_book_main}.
    \item \textbf{SLO (Service Level Objective):} Um objetivo de nível de serviço é um valor-alvo ou um intervalo de valores para um SLI, medido ao longo de um período \cite{mccoy_slo_2020}. Um SLO é a definição formal de "quão bom" o serviço precisa ser. Por exemplo, "95\% das requisições de leitura (SLI: latência de leitura) devem ser concluídas em menos de 250ms (SLO) nos últimos 28 dias".
    \item \textbf{Orçamento de Erro (Error Budget):} O orçamento de erro é o complemento do SLO (ou seja, $100\% - SLO\%$) \cite{google_sre_book_main}. Ele representa a quantidade de falhas "permitidas" (ex: requisições lentas ou com erro) durante o período. O orçamento de erro é uma ferramenta de gerenciamento: enquanto houver orçamento, a equipe de desenvolvimento tem "permissão" para lançar novas funcionalidades (que inerentemente trazem risco); se o orçamento se esgotar, o foco da equipe deve mudar para a melhoria da confiabilidade \cite{google_sre_book_main}.
\end{itemize}

Esses fundamentos orientam a abordagem metodológica detalhada na Seção \ref{sec:metodologia}, que explica como o planejamento dos SLOs, a infraestrutura como código e os experimentos com k6 e Cloud Tasks foram conduzidos para gerar as evidências analisadas posteriormente.

\section{Metodologia}
\label{sec:metodologia}

O estudo foi conduzido de ponta a ponta, do planejamento dos objetivos de nível de serviço (SLOs) à coleta e interpretação dos experimentos. O enfoque é aplicado e experimental: toda a instrumentação foi construída diretamente no repositório ValorizeAI, o que permite a reprodução dos resultados.

\subsection{Tipo de Pesquisa e Estratégia Geral}

O trabalho caracteriza-se como uma \textbf{pesquisa aplicada} conduzida como \textbf{estudo de caso} de um sistema real em produção. A estratégia seguiu quatro fases iterativas. No \textbf{planejamento}, foram definidos os SLOs (latência P95 de 250~ms, erro $<$0{,}5\%, disponibilidade $\geq$99{,}5\%), mapeadas as cotas vigentes do Cloud Run (10 instâncias de 1~vCPU / 1~GiB, totalizando 10 vCPU) e estimado como essa limitação poderia afetar o throughput — nos ensaios preliminares o workload saturou próximo de 900 RPS, valor usado apenas como referência empírica. Em seguida veio a \textbf{preparação do ambiente}: módulos Terraform provisionaram rede, bancos e serviços gerenciados; Docker Compose reproduziu localmente PostgreSQL, Redis e a stack de observabilidade; o Makefile encapsulou tarefas de lint, testes e execução dos cenários. Na etapa de \textbf{execução controlada}, os cenários k6 de leitura e leitura/escrita foram disparados contra a API em Cloud Run enquanto o pipeline assíncrono recebia um lote adicional de tarefas no Cloud Tasks, exercitando os workers HTTP. Por fim, na \textbf{coleta e análise}, as métricas agregadas (latência, throughput, taxa de erro) foram extraídas dos CSVs e painéis do Cloud Monitoring, e as observações qualitativas sobre o teste de filas foram registradas juntamente com o tempo total de drenagem, subsidiando os capítulos de implementação e resultados.

\subsection{Arquitetura do Ambiente Experimental}

A Figura \ref{fig:arquitetura} sintetiza os componentes usados nos experimentos. O tráfego HTTP/HTTPS entra por um \textbf{Cloud Load Balancer} com \textbf{Cloud CDN}, que reduz a latência de \textit{assets} estáticos e protege o backend com inspeção WAF. Esse tráfego é encaminhado para dois serviços Cloud Run:
\begin{itemize}
    \item \textbf{API Laravel}: processa requisições REST, expõe endpoints usados pelos testes k6 e orquestra o pipeline assíncrono.
    \item \textbf{Laravel Reverb}: mantém conexões WebSocket persistentes para eventos em tempo real; é tratado como serviço independente para permitir escalonamento específico.
\end{itemize}

Ambos os serviços acessam o \textbf{Memorystore for Redis}, usado simultaneamente como cache de leitura (padrão \textit{cache-aside}) e como \textit{backplane} Pub/Sub do Reverb. O armazenamento transacional permanece no \textbf{Cloud SQL for PostgreSQL}, que atende às operações de leitura e escrita executadas durante os testes. Para workloads assíncronos, a API publica tarefas em \textbf{Cloud Tasks}, que aciona workers HTTP também hospedados no Cloud Run. Artefatos grandes (extratos e relatórios) são persistidos no \textbf{Cloud Storage}, mas não fizeram parte dos testes de carga.

\begin{figure}[ht]
    \centering
    \resizebox{\linewidth}{!}{%
    \begin{tikzpicture}[node distance=1.5cm, every node/.style={font=\footnotesize, align=center}]
        \node (cdn) [draw, rounded corners, fill=gray!15, minimum width=5cm, minimum height=0.9cm] {Cloud Load Balancer + Cloud CDN};
        \node (api) [draw, rounded corners, fill=blue!10, minimum width=3cm, minimum height=0.9cm, below left=1.1cm and 2.0cm of cdn] {Cloud Run\\API};
        \node (reverb) [draw, rounded corners, fill=blue!10, minimum width=3cm, minimum height=0.9cm, below=1.1cm of cdn] {Cloud Run\\Reverb};
        \node (workers) [draw, rounded corners, fill=blue!10, minimum width=3cm, minimum height=0.9cm, below right=1.1cm and 2.0cm of cdn] {Cloud Run\\Workers};
        \node (shared) [draw, rounded corners, fill=orange!15, minimum width=6cm, minimum height=1.2cm, below=1.3cm of reverb] {Cloud SQL + Memorystore (Redis) + Cloud Storage};
        \node (tasks) [draw, rounded corners, fill=green!10, minimum width=5cm, minimum height=0.9cm, below=1.0cm of shared] {Cloud Tasks};

        \draw[->, thick] (cdn) -- (api);
        \draw[->, thick] (cdn) -- (reverb);
        \draw[->, thick] (cdn) -- (workers);
        \draw[->, thick] (api) -- (shared);
        \draw[->, thick] (reverb) -- (shared);
        \draw[->, thick] (workers) -- (shared);
        \draw[->, thick] (api) |- (tasks);
        \draw[->, thick] (tasks) -| (workers);
    \end{tikzpicture}}
    \caption{Arquitetura utilizada nos experimentos.}
    \label{fig:arquitetura}
\end{figure}

\subsection{Ferramentas e Processo de Preparação}

Do ponto de vista de engenharia, três pilares garantiram a reprodutibilidade:
\textbf{(i)} \emph{Infraestrutura como Código}: os módulos Terraform descrevem VPC, balanceadores, Cloud Run, Cloud SQL, Redis e Cloud Tasks. Cada mudança passa por \textit{plan/apply} versionado, evitando deriva de ambiente.
\textbf{(ii) Ambientes determinísticos}: o Makefile e os manifestos Docker recompõem o stack local (PostgreSQL, Redis, Loki/Tempo e PHP 8.4) idêntico ao ambiente de teste antes de qualquer execução k6.
\textbf{(iii) Observabilidade}: OpenTelemetry + Cloud Monitoring coletam métricas de latência, uso CPU/memória e backlog de filas, permitindo correlacionar cada rodada com os SLOs definidos.

\subsection{Planejamento dos SLOs e Desenho dos Cenários}

Com base nas premissas de negócio e na literatura de SRE \cite{mccoy_slo_2020,google_sre_book_main}, o sistema foi avaliado contra três metas: latência P95 $\leq 250$~ms, taxa de erro $<$ 0{,}5\% e disponibilidade mensal $\geq 99{,}5\%$. A cota vigente do Cloud Run (10 instâncias de 1~vCPU/1~GiB) limita o total de CPU disponível; no nosso cenário isso significou que os testes deveriam aumentar a carga até consumir essas 10 vCPU (o que, empiricamente, ocorreu perto de 900 RPS), documentando o comportamento imediatamente antes do esgotamento.

Dois cenários foram modelados:
\begin{enumerate}
    \item \textbf{Leitura intensiva}: 1{.}000 usuários virtuais consultando listas de transações por 17 minutos em seis estágios, exercitando cache Redis + réplica de leitura do PostgreSQL.
    \item \textbf{Mistura leitura/escrita}: 650 usuários virtuais alternando consultas e criação de transações durante 21 minutos, forçando locks no banco e pressionando o pipeline de escrita.
\end{enumerate}
Além desses ensaios HTTP, foi planejado um \textbf{teste de filas} no qual um volume elevado de tarefas artificiais percorre o fluxo Cloud Tasks → workers HTTP, permitindo observar o tempo de drenagem e a elasticidade dos consumidores assíncronos.

\subsection{Execução dos Experimentos}

Cada rodada segue os passos:
\begin{enumerate}
    \item \textbf{Preparação dos dados}: seeds e factories povoam o PostgreSQL com contas, transações e orçamentos realistas; a instância Redis é pre-aquecida com métricas e dashboards frequentes.
    \item \textbf{Disparo do cenário}: os perfis do k6 focados em leitura e no mix leitura/escrita são executados via Makefile, apontando para o domínio público do Cloud Load Balancer; estágios, VUs e SLIs monitorados seguem o planejamento experimental.
    \item \textbf{Registro automático}: os resultados agregados são gravados em CSVs (latência, taxa de erro, uso de VUs) e correlacionados com as métricas de infraestrutura capturadas pelo Cloud Monitoring.
    \item \textbf{Teste de filas}: um script HTTP produz um lote adicional de tarefas e a drenagem é acompanhada por meio das métricas do Cloud Tasks e dos logs dos workers.
\end{enumerate}

\subsection{Coleta e Integração das Evidências}

As evidências produzidas sustentam as análises de arquitetura, implementação e resultados:
\begin{itemize}
    \item \textbf{Planilhas de latência e throughput}: derivadas dos CSVs exportados pelo k6, utilizadas posteriormente para comparar as métricas observadas com os SLOs.
    \item \textbf{Series temporais de infraestrutura}: capturas dos dashboards do Cloud Monitoring registram uso de CPU das instâncias Cloud Run, saturação do Redis e backlog do Cloud Tasks durante cada rodada.
    \item \textbf{Relatos de execução}: cada rodada é registrada em um diário experimental com horários, parâmetros e observações qualitativas sobre o comportamento do sistema.
\end{itemize}

Essa metodologia garante rastreabilidade completa entre arquitetura, implementação e resultados, pois cada passo experimental está ancorado em artefatos versionados do projeto.

\section{Resultados e Discussão}
\label{sec:resultados}

Esta seção consolida as métricas obtidas nos testes de carga (k6) e no ensaio assíncrono com Cloud Tasks. As medições seguem os SLIs definidos na metodologia: latência P95, taxa de erro e tempo de drenagem do pipeline assíncrono.

\begin{table}[ht]
    \centering
    \caption{Resumo dos cenários de carga e conformidade com os SLOs}
    \label{tab:resultados_k6}
    \begin{tabular}{lccc}
        \toprule
        Cenário & Latência P95 & Throughput médio & Taxa de erro \\
        \midrule
        Leitura intensiva & 158~ms & 470 req/s & 0{,}00\% \\
        Mistura leitura/escrita & 658~ms & 226 req/s & 0{,}00\% \\
        \bottomrule
    \end{tabular}
\end{table}

Os valores de \textit{throughput} foram obtidos diretamente dos dashboards do Cloud Run para cada estágio de VUs; por exemplo, o platô de 650 VUs no cenário misto correspondeu a aproximadamente 226 requisições por segundo, enquanto o patamar final de 900→1000 VUs no cenário de leitura atingiu pico de cerca de 970 req/s (a média global do teste ficou em 470 req/s). Assim, VUs e req/s descrevem o mesmo nível de pressão gerada nos serviços.

\subsection{Cenário de Leitura Intensiva}

Os 1.000 usuários virtuais do cenário de leitura mantiveram o backend em média de 470 requisições por segundo durante 17 minutos, com pico aproximado de 970 req/s no platô final (900→1000 VUs). A latência P95 permaneceu em 158~ms (Tabela \ref{tab:resultados_k6}), confortavelmente dentro do SLO de 300~ms mesmo nesse pico, e nenhuma requisição falhou. Isso confirma que o caminho otimizado para consultas --- balanceador/CDN, API em Cloud Run, cache Redis e leituras do PostgreSQL --- absorve picos de leitura sem degradação perceptível. As métricas do Cloud Monitoring confirmaram o escalonamento até as 10 instâncias de Cloud Run disponíveis, com CPU média de 72\% e memória de 31\%, mostrando que a cota foi integralmente utilizada sem comprometer a latência.

\begin{figure}[ht]
    \centering
    \includegraphics[width=0.95\linewidth]{figures/fig-read-pxx.png}
    \caption{Latências P50/P95/P99 por carga (VUs) no cenário de leitura. A linha tracejada indica o SLO de 300~ms, não violado durante o teste.}
    \label{fig:latencia-leitura}
\end{figure}

\subsection{Cenário Misto (Leitura/Escrita)}

O cenário misto foi executado com provisionamento automático de usuários/token por VU, eliminando o gargalo artificial observado quando todos competiam pela mesma conta de teste.
Com 650 usuários alternando entre 65\% de leituras e 35\% de escritas durante 21 minutos, o sistema sustentou 226 requisições por segundo (283~mil iterações) sem erros e preservou o SLO de 300~ms até aproximadamente 450 RPS (a violação é observada ao redor de 539 VUs).
Assim que a carga ultrapassou 550 VUs, o Cloud Run atingiu novamente o teto de 10 instâncias e o P95 consolidado do ensaio ficou em 658~ms (p99 = 2,67~s). A pressão veio do aumento de CPU e latência nas operações de escrita, já que cada instância precisou executar validações, persistência e invalidação de cache antes de responder; não houve evidência de saturação no Cloud SQL, cujos tempos de consulta permaneceram estáveis durante todo o gradiente.
Esse comportamento indica que, antes de pensar em particionamento de dados, é necessário permitir que mais instâncias HTTP sejam acionadas ou otimizar o custo das rotas de escrita para consumir menos CPU por requisição. Também ficou evidente o benefício do endpoint de provisionamento: cada VU pôde criar contas próprias e evitar competição por saldo, aproximando o teste de um uso real.

\begin{figure}[ht]
    \centering
    \includegraphics[width=0.95\linewidth]{figures/fig-mixed-pxx.png}
    \caption{Latências P50/P95/P99 por carga (VUs) no cenário misto atualizado. A faixa tracejada marca o SLO de 300~ms; as etapas acima de 539~VUs apresentam violações persistentes desse SLO.}
    \label{fig:latencia-mix}
\end{figure}

\begin{figure}[ht]
    \centering
    \includegraphics[width=0.95\linewidth]{figures/fig-compare-read-vs-mixed.png}
    \caption{Comparação direta entre os P95 das cargas de leitura e mista. O cenário misto viola o SLO aos 539~VUs, enquanto a curva de leitura permanece abaixo de 200~ms mesmo com 1.000~VUs.}
    \label{fig:comparacao}
\end{figure}

\noindent A Figura~\ref{fig:comparacao} evidencia o distanciamento entre os dois perfis: na leitura os percentis crescem suavemente até o pico, sinalizando que o gargalo ainda está restrito à cota de instâncias; já no misto, a curva dispara logo após 500~VUs porque cada POST exige mais trabalho por instância (validações, escrita e invalidação de cache), reduzindo o número de requisições atendidas por contêiner quando o limite de 10 instâncias é atingido. Esse comportamento é consistente com a literatura sobre sistemas transacionais centralizados \cite{kleppmann_ddia_2017}, que destaca a dificuldade de um banco relacional único absorver altas taxas de escrita e sincronização sem ajustes adicionais. Assim, a arquitetura atual sustenta leituras em larga escala, mas depende de mais instâncias HTTP (ou otimizações focadas em escrita) antes de considerar estratégias mais complexas de particionamento ou réplicas.

\subsection{Processamento Assíncrono com Cloud Tasks}

O ensaio de filas injetou 51{,}58~mil tarefas em um único lote e monitorou a drenagem entre 01:08 e 01:18, totalizando aproximadamente 10~min. Isso corresponde a um ritmo médio de 86 tarefas/segundo por worker HTTP em Cloud Run, comprovando que a combinação Cloud Tasks + workers escala horizontalmente sem perder entregas: cada aumento de backlog gerou novos \textit{containers} apenas durante a janela de pico, sem repetição indevida ou perda de tarefas. A latência ponta-a-ponta permaneceu estável porque o Redis atuou somente como \textit{backplane} para eventos, reduzindo contenção. Como Cloud Run cobra por instância ativa e Cloud Tasks por requisições enfileiradas, a elasticidade observada evita o superprovisionamento típico de VMs fixas: as 10 instâncias só permaneceram ativas enquanto havia tarefas, o que, mesmo sem detalhar faturamento, sugere controle de custo alinhado à carga real. Esse resultado valida que o pipeline assíncrono consegue absorver rajadas equivalentes ao dobro do tráfego diário da aplicação sem intervenção manual.

\begin{figure}[ht]
    \centering
    \includegraphics[width=0.95\linewidth]{figures/fig-queue-drain.png}
    \caption{Taxa de processamento das filas (tarefas/min) e janela de drenagem entre 01:08 e 01:18 no ensaio com Cloud Tasks.}
    \label{fig:filas}
\end{figure}

\section{Conclusão}

\subsection{Principais Contribuições}

O ValorizeAI demonstrou que é possível documentar e validar, de ponta a ponta, uma arquitetura orientada a eventos construída exclusivamente com serviços gerenciados. O trabalho conecta decisões de design (Cloud Run, Redis, Cloud SQL, Cloud Tasks, Reverb) ao domínio transacional modelado no banco relacional, oferecendo um roteiro explícito para equipes que precisam de elasticidade sem perder consistência forte. A metodologia experimental, baseada em SRE, complementa essa documentação ao detalhar como planejar SLOs, materializar infraestrutura como código e acionar testes repetíveis, fortalecendo a reprodutibilidade típica de uma pesquisa aplicada.

\subsection{Resultados Alcançados}

Os cenários de carga comprovaram a eficácia das otimizações de leitura: com 1.000 usuários virtuais ($\approx$470 req/s em média e pico de $\approx$970 req/s no platô final), o sistema manteve latência P95 em 158~ms e zero erros, satisfazendo os SLOs estabelecidos. O cenário misto, por sua vez, revelou o limiar atual da solução; o patamar de 650 VUs ($\approx$226 req/s) levou o P95 a 658~ms e o p99 a 2,67~s porque o Cloud Run estacionou no limite de 10 instâncias e cada requisição de escrita consome mais CPU devido aos passos adicionais (validações, persistência, invalidação de cache). O Cloud SQL manteve tempos estáveis, indicando que o gargalo imediato ainda é computacional na camada HTTP. No plano assíncrono, a drenagem de 51{,}58~mil tarefas em 10~min confirmou que Cloud Tasks e workers em Cloud Run sustentam bursts significativos sem intervenção humana, atendendo aos requisitos de elasticidade e rastreabilidade.

\subsection{Limitações}

Dois fatores limitam os resultados. Primeiro, o ambiente operou com as cotas padrão de uma conta recém-provisionada (10 instâncias de 1~vCPU), o que restringe a observação de comportamentos em estágios superiores de escala. Segundo, o banco de dados permaneceu concentrado em uma única instância Cloud SQL regional; embora consistente, essa configuração impõe contenções em workloads que concentram escritas nas mesmas contas ou categorias.

\subsection{Trabalhos Futuros}

As próximas etapas incluem ampliar gradualmente a cota de Cloud Run para liberar mais instâncias e repetir os testes sob 20 ou 40~vCPU, ao mesmo tempo em que se avaliam os próximos gargalos: réplicas de leitura do Cloud SQL para aliviar consultas analíticas, escalonamento vertical das instâncias primárias para absorver picos de escrita e mecanismos de \textit{failover} multi-região para banco e Redis. Também se pretende complementar o pipeline assíncrono com um orquestrador baseado em eventos (ex.: Workflows) para medir a resiliência a falhas parciais, e estender o uso de embeddings e automações de categorização para outras áreas (como detecção de anomalias).

\newpage

% \nocite{BCB_Economia_Bancaria_2024,Febraban_Tecnologia_2024,Febraban_Transacoes_2025}

\bibliographystyle{article-config/sbc}
\bibliography{references/references}

\end{document}
