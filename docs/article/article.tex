%%%%%%%%%%%%%%%%%%%%%%%%%%%%%%%%%%%%%%%%%%%%%%%%%%%%%%%%%%%%%%%%%%%%%%
% How to use writeLaTeX:
%
% You edit the source code here on the left, and the preview on the
% right shows you the result within a few seconds.
%
% Bookmark this page and share the URL with your co-authors. They can
% edit at the same time!
%
% You can upload figures, bibliographies, custom classes and
% styles using the files menu.
%
% Modifyied by Prof. MSc. Daniel Menin Tortelli
% Computer Science - URICER - Brazil
%%%%%%%%%%%%%%%%%%%%%%%%%%%%%%%%%%%%%%%%%%%%%%%%%%%%%%%%%%%%%%%%%%%%%%

\documentclass[12pt]{article}

\usepackage{article-config/sbc-template}

\usepackage{graphicx,url}

\usepackage[brazilian]{babel}
\usepackage[T1]{fontenc}
\usepackage[utf8]{inputenc}
\usepackage{multicol}
\usepackage{graphicx,url}
\usepackage{amsmath,amssymb,amsfonts}
\usepackage{algorithmic}
\usepackage{textcomp}
\usepackage{url}
\usepackage{enumerate}
\usepackage{multirow}
\usepackage{soul}
\usepackage{verbatim}
\usepackage[table,xcdraw]{xcolor}
\usepackage[shortlabels]{enumitem}
\usepackage{scalefnt}
\usepackage{comment}
\usepackage{lscape}
\usepackage{fancyhdr}
\usepackage{article-config/pages-style}
\usepackage[super]{nth}
\usepackage[table]{xcolor}
\usepackage{longtable}
\usepackage{booktabs}
\usepackage{tikz}
\usepackage{pgfplots}
\pgfplotsset{compat=1.18}
\usetikzlibrary{positioning,calc}
\usepackage{float}


\tolerance=1
\emergencystretch=\maxdimen
\hyphenpenalty=10000
\hbadness=10000

\sloppy


\curso{Ciência da Computação} % Preencha com o nome do curso
\local{URI Erechim/RS} % Local do evento
\ano{2025 } % Exemplo: 2025
\edicao{} % Exemplo: XXIII

\title{ValorizeAI: Documentação e Validação de uma Arquitetura Serverless Elasticamente Gerenciada \\
\medskip
\small{
    \textit{Title: ValorizeAI: Documenting and Validating a Managed Serverless Architecture}
}
}

\author{Felipe Tomkiel Malacarne, Prof. Me. Marcos André Lucas}

\address{
    Universidade Regional Integrada do Alto Uruguai e das Missões \\
    Departamento de Engenharias e Ciência da Computação\\
  Caixa Postal 743 -- 99.709-910 -- Erechim -- RS -- Brasil
\email{101090@uricer.edu.br, mlucas@uricer.edu.br}
}

\begin{document}

\maketitle

\thispagestyle{plain}
\pagestyle{plain2}


\begin{resumo}
Plataformas financeiras modernas exigem arquiteturas elásticas capazes de absorver picos de carga transacional sem comprometer consistência forte ou observabilidade. Este trabalho documenta e valida o ValorizeAI, uma aplicação real de categorização automática construída sobre Google Cloud Run, Cloud SQL, Redis (Memorystore), Cloud Tasks e WebSockets. A pesquisa adota uma abordagem aplicada inspirada em SRE: os SLOs (latência P95 $\leq$ 300~ms, erro $<$ 0,5\%, disponibilidade $\geq$ 99{,}5\%) são definidos previamente, a infraestrutura é provisionada como código e os testes k6 abrangem cenários de leitura intensiva, leitura/escrita e o pipeline assíncrono. O cenário de leitura sustentou 1{,}000 usuários virtuais (média de 470 req/s, pico de $\approx$970 req/s) com P95 de 158~ms; o cenário misto manteve 226 req/s com 650 VUs e violou o SLO a partir de $\approx$539 VUs, elevando o P95 para 658~ms e o p99 para 2,67~s ao saturar a cota de 10 instâncias do Cloud Run. O pipeline assíncrono processou 51,58 mil tarefas em 10 minutos (86 tarefas/s) sem perdas. Conclui-se que a arquitetura atende confortavelmente workloads intensivos em leitura e que o suporte a escritas altamente concorrentes requer aumento de instâncias HTTP ou otimizações específicas no caminho de escrita.
\end{resumo}

\palavraschave{Computação Serverless, Cloud Run, Plataformas Financeiras, Testes de Carga, Observabilidade.}

\begin{abstract}
Modern financial workloads demand elastic architectures capable of sustaining variable traffic while preserving strong consistency and system observability. This paper documents and validates ValorizeAI, a real-world automated categorization platform built on Cloud Run, Cloud SQL, Redis, Cloud Tasks, and WebSockets. The study adopts an applied SRE-inspired methodology: SLOs (P95 latency $\leq$ 300~ms, error rate $<$ 0.5\%, availability $\geq$ 99.5\%) are defined beforehand, infrastructure is provisioned as code, and k6 executes read-only, mixed read/write, and asynchronous-pipeline load scenarios. The read scenario supported 1,000 virtual users (mean 470 req/s, peak $\approx$970 req/s) with P95 = 158~ms and zero failures; the mixed scenario sustained 226 req/s at 650 VUs and violated the SLO around $\approx$539 VUs, reaching P95 = 658~ms and p99 = 2.67~s once the 10-instance Cloud Run quota was saturated. The asynchronous pipeline processed 51.58k Cloud Tasks in 10 minutes (86 tasks/s) with no loss or duplication. The results show that the architecture comfortably supports read-heavy traffic, while higher write concurrency requires additional horizontal scaling or optimizations in the write path.
\end{abstract}

\keywords{Serverless Computing, Cloud Run, Financial Platforms, Performance Testing, Observability.}

\section{Introdução}
\label{sec:introducao}

O ValorizeAI, objeto deste estudo, surge como um caso completo para investigar a relação entre elasticidade e observabilidade. Trata-se de uma aplicação web modular que integra ingestão de dados, processamento síncrono e assíncrono e comunicação em tempo real, utilizando uma \textit{stack} moderna baseada em Cloud Run, Cloud SQL, Redis (Memorystore), Cloud Tasks, WebSockets e serviços auxiliares.

Aplicações digitais modernas — incluindo plataformas de e-commerce, serviços financeiros e sistemas colaborativos — enfrentam o desafio de absorver cargas de trabalho voláteis sem comprometer a experiência do usuário \cite{google_elasticity_2024}. A resposta predominante tem sido a \textit{elasticidade na nuvem}, isto é, a capacidade de alocar e desalocar recursos automaticamente conforme a demanda varia, substituindo o provisionamento manual por escalonamento em tempo real para evitar desperdícios e manter a responsividade.

Essa elasticidade, implementada com microsserviços, contêineres e paradigmas \textit{serverless}, introduz complexidade que só pode ser controlada com observabilidade avançada \cite{newrelic_observability_2023}. Logs, métricas e \textit{traces} tornam-se insumos para SLIs que, por sua vez, validam e ajustam políticas automáticas de escalonamento. Elasticidade e observabilidade formam, assim, um ciclo de \textit{feedback} que opera em horizontes temporais inviáveis para processos manuais, reforçando a necessidade de arquiteturas e práticas integradas.

\subsection{Justificativa e Problema de Pesquisa}
\label{sec:justificativa}

Workloads transacionais que envolvem ingestão intensa de dados, estados compartilhados e interfaces colaborativas, como os simulados pelo ValorizeAI, impõem requisitos rigorosos: consistência forte, rastreabilidade para auditoria e respostas de baixa latência mesmo sob variações abruptas de tráfego. Para atender a esses requisitos, arquiteturas modernas combinam componentes especializados, como CDNs e balanceadores globais, filas assíncronas orientadas a eventos, cache distribuído e comunicação persistente via WebSockets \cite{barri_scalability_2025, confluent_eda_2024, yadav_redis_leaderboard_2019, fernando_websocket_2025}.

Embora existam estudos pontuais sobre esses componentes, a literatura apresenta lacunas quanto à validação integrada de arquiteturas híbridas (CaaS + filas + WebSockets) em cenários reprodutíveis de carga \cite{wjaets_serverless_ml_2022, abad_serverless_gap_2021}. Trabalhos existentes tendem a focar em comparações de ferramentas de IaC \cite{pessa_iac_2023} ou no desempenho de microsserviços isolados \cite{hebbar_reactive_2025}, raramente considerando o comportamento do sistema completo.

Este trabalho busca preencher essa lacuna ao documentar a arquitetura do ValorizeAI e validar seu comportamento sob estresse de carga frente a SLOs definidos. A questão central investigada é: \textit{Como uma arquitetura híbrida e elástica, composta por Cloud Run, Redis, Cloud SQL, Cloud Tasks e WebSockets dedicados, se comporta sob condições intensas de carga, e como esse comportamento pode ser validado de forma reprodutível?}

\subsection{Objetivo Geral}
\label{sec:objetivo_geral}

Demonstrar, por meio de documentação técnica e experimentos de desempenho, que a arquitetura do ValorizeAI, composta por CDN, contêineres escalados horizontalmente, processamento assíncrono em filas, servidor de WebSockets, armazenamento de artefatos em \textit{buckets} e cache distribuído em Redis, sustenta os SLOs definidos para um produto transacional completo, mantendo código, infraestrutura e observabilidade versionados em repositório.

\subsection{Objetivos Específicos}
\label{sec:objetivos_especificos}

\begin{enumerate}
    \item Mapear a arquitetura \textit{end-to-end}, destacando o papel do balanceador/CDN, instâncias de contêineres, servidor WebSockets, filas assíncronas, \textit{buckets} e Redis.
    \item Documentar o desenvolvimento dos módulos críticos do sistema, incluindo ingestão de dados, automações, notificações e painéis em tempo real.
    \item Planejar e executar testes de carga (k6, cenários de leitura e leitura/escrita) e testes assíncronos, validando a arquitetura frente aos SLOs definidos.
    \item Interpretar os resultados e propor otimizações relacionadas a desempenho, elasticidade e custo.
\end{enumerate}

\subsection{Contribuições Tangíveis}
\label{sec:contribuicoes}

\begin{enumerate}
    \item Arquitetura documentada e replicável.
    \item Infraestrutura reprodutível (Terraform, Docker, Makefile).
    \item Cenários de desempenho registrados e transparentes (k6).
    \item Validação do pipeline assíncrono baseado em Cloud Tasks.
\end{enumerate}

Em conjunto, essas contribuições formam um pacote completo de replicação, código, automações e experimentos, para avaliações futuras de workloads transacionais em ambientes CaaS.

\vspace{0.5em}
O restante deste artigo está organizado da seguinte forma:
a Seção~\ref{sec:relacionados} apresenta os trabalhos relacionados;
a Seção~\ref{sec:fundamentacao} discute a fundamentação teórica;
a Seção~\ref{sec:metodologia} descreve a metodologia experimental;
a Seção~\ref{sec:implementacao} detalha a implementação do ValorizeAI;
a Seção~\ref{sec:resultados} consolida os resultados e análises;
e, por fim, a Seção~\ref{sec:conclusao} apresenta as conclusões e trabalhos futuros.

\section{Trabalhos Relacionados}
\label{sec:relacionados}

A arquitetura investigada neste trabalho situa-se na interseção de três eixos centrais da literatura em sistemas distribuídos:
(i) paradigmas de execução em nuvem,
(ii) padrões arquiteturais para desempenho e resiliência e
(iii) metodologias de validação empírica.
Esta seção revisa o estado da arte nesses eixos para posicionar a contribuição do ValorizeAI e evidenciar a lacuna identificada na seção de Introdução.

\subsection{Paradigmas de Execução: Serverless (FaaS) e Contêineres Gerenciados (CaaS)}
\label{sec:paradigmas}

A escolha do paradigma de execução é um dos fatores determinantes para sistemas elásticos modernos. A literatura recente compara amplamente \textit{Functions-as-a-Service} (FaaS) e \textit{Containers-as-a-Service} (CaaS). O FaaS, amplamente representado por AWS Lambda e Google Cloud Functions, oferece escalonamento automático e faturamento por execução, mas apresenta limitações para aplicações \textit{stateful} ou de longa duração devido ao \textit{cold start}, ao isolamento elevado e à efemeridade das instâncias \cite{sonawane_serverless_review_2024, datadog_serverless_containers_2024}. Esses aspectos o tornam inadequado para componentes persistentes como servidores WebSocket.

Por outro lado, o CaaS — como Google Cloud Run ou AWS Fargate — mantém a elasticidade do FaaS, mas preserva o controle sobre o ambiente do contêiner e comporta processos contínuos \cite{lloyd_serverless_investigation_2018}. Essa característica é essencial para o servidor de WebSockets do ValorizeAI (Laravel Reverb), que requer conexões persistentes e compartilhamento de estado via Redis. Assim, a literatura sustenta a escolha do CaaS como paradigma mais adequado para arquiteturas híbridas que combinam serviços \textit{stateless} e componentes de longa duração.

\subsection{Padrões Arquiteturais para Desempenho e Resiliência}
\label{sec:padroes}

Para que a arquitetura do ValorizeAI atenda aos SLOs definidos,
ela precisa combinar padrões síncronos e assíncronos capazes de manter responsividade,
disponibilidade e escalabilidade diante de variações de carga.
A literatura descreve esses padrões como peças complementares, filas assíncronas absorvem rajadas,
enquanto canais em tempo real sustentam interações colaborativas, e fornece referências diretas para as escolhas arquiteturais adotadas neste trabalho.

\subsubsection{Arquiteturas Orientadas a Eventos (EDA) e Filas Assíncronas}
\label{sec:eda}

Arquiteturas orientadas a eventos promovem desacoplamento, resiliência e absorção de picos de carga ao delegar tarefas para filas assíncronas \cite{confluent_eda_2024}. Estudos comparativos analisam o comportamento de diferentes \textit{brokers} — como RabbitMQ, Apache Kafka e Pulsar — frente a cenários com variação de tamanho de mensagens e taxas de publicação \cite{thepphakan_pulsar_rabbitmq_2025}. Esses trabalhos demonstram que a escolha da fila deve refletir o perfil do \textit{workload}: baixa latência para eventos pequenos ou alto \textit{throughput} contínuo para fluxos intensivos.

A arquitetura do ValorizeAI segue essa abordagem ao integrar Cloud Tasks para desacoplar operações pesadas do ciclo de requisição HTTP, garantindo responsividade mesmo sob carga.

\subsubsection{Comunicação em Tempo Real e Cache Distribuído}
\label{sec:realtime}

Para requisitos de tempo real, a literatura destaca o uso combinado de WebSockets e cache distribuído. Redis é amplamente utilizado como memória compartilhada de baixa latência e como mecanismo Pub/Sub para orquestrar a entrega de eventos entre múltiplas instâncias \cite{yadav_redis_leaderboard_2019}. Em ambientes elásticos — como Cloud Run — o Redis atua como \textit{backplane} que mantém a consistência entre instâncias efêmeras ao distribuir mensagens para clientes conectados.

Estudos recentes analisam o impacto desse modelo em métricas de \textit{throughput} e latência RTT, validando suas vantagens para sistemas colaborativos e dashboards em tempo real \cite{fernando_websocket_2025, twine_websocket_scaling_2022}. Esse padrão é consistente com o design do ValorizeAI, cujo servidor WebSocket persistente publica e assina eventos via Redis para garantir consistência e escalabilidade horizontal.

\subsection{Metodologias de Validação Empírica}
\label{sec:metodologias_validacao}

Validar arquiteturas distribuídas exige metodologias reprodutíveis e alinhadas a objetivos de serviço.

\subsubsection{Infraestrutura como Código (IaC)}
\label{sec:iac}

IaC é amplamente adotado para garantir reprodutibilidade e eliminar variações de ambiente em experimentos de desempenho \cite{pessa_iac_2023}. Estudos analisam ferramentas como Terraform e AWS CDK em termos de eficiência, expressividade e redução de deriva de configuração \cite{guerriero_iac_2019}. Entretanto, tais estudos frequentemente focam na ferramenta, e não no sistema provisionado — o que contrasta com este trabalho, no qual IaC é utilizado como base para experimentos e revalidações sucessivas do ambiente completo (CaaS, filas, Redis e banco de dados).

\subsubsection{Validação por SLOs e Testes de Carga}
\label{sec:k6_slo}

A Engenharia de Confiabilidade de Sites (SRE) recomenda validação orientada a SLOs para medir sucesso operacional \cite{mccoy_slo_2020}. A literatura recente adota ferramentas modernas como o k6 para simular perfis realistas de carga e avaliar latência, erro e saturação \cite{cervone_k6_2024}. O trabalho de Hebbar \cite{hebbar_reactive_2025}, por exemplo, utiliza k6 para validar priorização de tráfego em APIs reativas, monitorando percentis de latência e comportamento sob estresse.

A estratégia utilizada pelo ValorizeAI — definição de SLOs, instrumentação do sistema e execução de cenários de carga reprodutíveis — encontra suporte direto nesses estudos, reforçando a adequação da metodologia adotada.

\subsection{Síntese e Lacuna de Pesquisa}
\label{sec:lacuna}

A literatura revisada é rica, mas fragmentada: há estudos sobre FaaS vs. CaaS \cite{lloyd_serverless_investigation_2018}, comparações de \textit{brokers} de EDA \cite{thepphakan_pulsar_rabbitmq_2025}, análises de ferramentas de IaC \cite{pessa_iac_2023} e validações de desempenho específicas usando k6 e SLOs \cite{hebbar_reactive_2025}. Contudo, conforme discutido por Abad et al. \cite{abad_serverless_gap_2021}, falta uma análise integrada de arquiteturas híbridas que combinem todos esses elementos em um único sistema reprodutível.

A contribuição do ValorizeAI está exatamente nessa integração: uma arquitetura CaaS orientada a eventos com WebSockets persistentes, cache Redis e filas assíncronas, provisionada integralmente via IaC e validada com metodologia alinhada ao estado da arte. Os estudos analisados servem como blocos isolados; este trabalho, por sua vez, propõe uma validação \textit{end-to-end} que abrange paradigma, padrões arquiteturais e metodologia experimental.

\section{Fundamentação Teórica}
\label{sec:fundamentacao}

Segue uma síntese do vocabulário e das bases conceituais utilizados no design, implementação e validação do ValorizeAI, cobrindo princípios de design de software, arquitetura dos componentes e fundamentos de engenharia de confiabilidade.

\subsection{Princípios de Design de Software}
\label{sec:principios_design}

O ValorizeAI adota uma abordagem de "arquitetura limpa", segregando responsabilidades com base em princípios estabelecidos de design de software.

\subsubsection{Clean Architecture}
\label{sec:clean_architecture}

Formalizada por Robert C. Martin, a \textit{Clean Architecture} (Arquitetura Limpa) é um modelo arquitetural que advoga pela separação de interesses \cite{martin_clean_2017}. Seu objetivo é criar sistemas que sejam: (1) Independentes de frameworks; (2) Testáveis; (3) Independentes da interface do usuário (UI); e (4) Independentes do banco de dados \cite{martin_clean_2017}.

O pilar central dessa arquitetura é a \textit{Regra da Dependência} (The Dependency Rule). Esta regra estipula que as dependências do código-fonte devem apontar exclusivamente "para dentro" --- de camadas de baixo nível (detalhes voláteis, como frameworks e bancos de dados) para camadas de alto nível (políticas de negócio estáveis e abstrações) \cite{martin_clean_2017}. No ValorizeAI, isso se manifesta na separação das regras de negócio (localizadas em \textit{Actions} ou \textit{Queries}) da lógica do framework (Controladores Laravel) ou da persistência (Modelos Eloquent).

\subsubsection{Domain-Driven Design (DDD)}
\label{sec:ddd}

O \textit{Domain-Driven Design} (DDD), introduzido por Eric Evans, é uma abordagem para o desenvolvimento de software que se concentra em modelar o software para corresponder a um domínio de negócio complexo \cite{evans_ddd_2003}. O DDD é essencial para gerenciar a complexidade em sistemas como o ValorizeAI. Os conceitos-chave utilizados neste trabalho incluem:

\begin{itemize}
    \item \textbf{Linguagem Ubíqua (Ubiquitous Language):} Um vocabulário compartilhado e rigoroso, desenvolvido em colaboração entre os desenvolvedores e os especialistas do domínio (usuários). Essa linguagem é usada em todas as comunicações e reflete-se diretamente no código (nomes de classes, métodos e variáveis) \cite{evans_ddd_2003}.
    \item \textbf{Contexto Delimitado (Bounded Context):} A fronteira explícita dentro da qual um modelo de domínio e sua Linguagem Ubíqua são aplicáveis e consistentes \cite{evans_ddd_2003}.
    \item \textbf{Agregado (Aggregate):} Um cluster de objetos de domínio (Entidades e Objetos de Valor) que é tratado como uma única unidade para fins de consistência de dados. Um Agregado possui uma raiz (a \textit{Aggregate Root}), que é o único ponto de entrada para modificações dentro do Agregado, garantindo que todas as regras de negócio (invariantes) sejam aplicadas \cite{evans_ddd_2003}.
\end{itemize}

\subsubsection{Padrões de Comunicação e Segregação}
\label{sec:cqrs_dto}

Para implementar a Regra da Dependência e gerenciar o fluxo de dados, o ValorizeAI utiliza padrões de segregação e transferência de dados.

\begin{itemize}
    \item \textbf{DTO (Data Transfer Object):} Conforme popularizado por Martin Fowler, um DTO é um objeto simples, sem comportamento, cujo único propósito é transferir dados entre subsistemas ou camadas \cite{fowler_peaa_2002}. Em arquiteturas distribuídas ou em camadas, os DTOs são usados para agregar múltiplas chamadas em uma única, reduzindo a latência da rede e desacoplando os modelos internos (domínio) dos modelos de visualização (API/UI).
    \item \textbf{CQRS (Command Query Responsibility Segregation):} Um padrão, descrito por Martin Fowler \cite{fowler_cqrs_2011} e Greg Young, que propõe a segregação dos modelos de dados e da lógica de aplicação em duas categorias: \textit{Commands} (operações que alteram o estado, ou seja, escritas) e \textit{Queries} (operações que leem o estado). O ValorizeAI adota esse princípio através da separação explícita de \textit{Actions} (Commands) e \textit{Queries} (Queries), permitindo otimizações distintas para os caminhos de escrita e leitura.
\end{itemize}

\subsection{Arquitetura e Componentes da Aplicação}
\label{sec:componentes_app}

A infraestrutura do ValorizeAI é composta por serviços gerenciados na nuvem, escolhidos por suas características de elasticidade e desempenho.

\subsubsection{Google Cloud Run e Cloud Tasks}
\label{sec:cloud_run}

O Google Cloud Run é uma plataforma de computação CaaS (Container-as-a-Service) totalmente gerenciada. Ele permite a execução de contêineres \textit{stateless} que escalam horizontalmente de forma automática, com a capacidade de escalar até zero instâncias quando não há tráfego, eliminando custos ociosos \cite{google_elasticity_2024}. O serviço foi escolhido por combinar a elasticidade típica de funções serverless com a flexibilidade dos contêineres, executando tanto os serviços web \textit{stateless} do ValorizeAI quanto o servidor \textit{stateful} de WebSockets.

O Google Cloud Tasks é o serviço de enfileiramento de tarefas gerenciado. Ele é usado para implementar o processamento assíncrono (EDA), permitindo que a aplicação principal (síncrona) enfileire tarefas de longa duração (ex: processamento de lotes) para execução em \textit{workers} separados, garantindo resiliência e baixa latência na resposta ao usuário.

\subsubsection{Laravel Reverb (WebSockets)}
\label{sec:reverb}

O Laravel Reverb é o servidor WebSocket oficial de primeira-parte para aplicações Laravel, projetado para comunicação em tempo real de alto desempenho \cite{laravel_reverb_docs_2025}. Ele utiliza o protocolo Pusher, integrando-se nativamente ao sistema de \textit{broadcasting} do Laravel para facilitar o envio de notificações \textit{push} aos clientes conectados.

A característica arquitetural mais importante do Reverb para este TCC é seu suporte à escalabilidade horizontal. Para operar em um ambiente elástico como o Cloud Run (com múltiplas instâncias de servidor), o Reverb utiliza um \textit{backplane} de mensagens, que no caso do ValorizeAI é implementado com o Redis (detalhado na subseção sobre cache) \cite{laravel_reverb_docs_2025}.

\subsubsection{Redis (Remote Dictionary Server)}
\label{sec:redis}

O Redis (Remote Dictionary Server) é um armazenamento de estrutura de dados em memória, de código aberto, usado como banco de dados, \textit{cache} e \textit{message broker} \cite{kleppmann_ddia_2017}. No contexto da arquitetura ValorizeAI, o Redis desempenha dois papéis críticos e distintos, ambos fundamentais para o desempenho do sistema:

\begin{enumerate}
    \item \textbf{Cache de Baixa Latência:} O Redis é usado como um \textit{cache} para dados frequentemente acessados (ex: painéis, dados de sessão). Sua operação em memória permite latências de leitura e escrita na ordem de submilissegundos, reduzindo drasticamente a carga sobre o banco de dados PostgreSQL e melhorando a responsividade das \textit{Queries} \cite{yadav_redis_leaderboard_2019}.
    \item \textbf{Backplane Pub/Sub:} O Redis fornece um mecanismo de Publicação/Subscrição (Pub/Sub) de alto desempenho. Este mecanismo é utilizado como o \textit{backplane} do Laravel Reverb. Quando uma instância do servidor (Instância A) precisa notificar um usuário que está conectado via WebSocket a outra instância (Instância B), a Instância A publica a mensagem em um canal Redis. Todas as outras instâncias, incluindo a Instância B, estão inscritas nesse canal, recebem a mensagem e a retransmitem aos seus clientes WebSocket conectados localmente.
\end{enumerate}

\subsection{Engenharia de Confiabilidade de Sites (SRE)}
\label{sec:sre}

A metodologia de validação deste trabalho é baseada nos princípios de Engenharia de Confiabilidade de Sites (SRE), popularizados pelo Google \cite{google_sre_book_main}. O SRE trata as operações de infraestrutura como um problema de engenharia de software, utilizando métricas rigorosas para equilibrar a inovação (velocidade de desenvolvimento) com a confiabilidade do serviço.

\subsubsection{SLIs, SLOs e Orçamentos de Erro}
\label{sec:slo_sli}

Os conceitos centrais do SRE utilizados para a validação do ValorizeAI são:

\begin{itemize}
    \item \textbf{SLI (Service Level Indicator):} Um indicador de nível de serviço é uma medida quantitativa de um aspecto da qualidade do serviço fornecido \cite{mccoy_slo_2020}. Os SLIs são métricas diretas do desempenho do sistema, como latência de requisição, taxa de erro ou \textit{throughput} do sistema \cite{google_sre_book_main}.
    \item \textbf{SLO (Service Level Objective):} Um objetivo de nível de serviço é um valor-alvo ou um intervalo de valores para um SLI, medido ao longo de um período \cite{mccoy_slo_2020}. Um SLO é a definição formal de "quão bom" o serviço precisa ser. Por exemplo, "95\% das requisições de leitura (SLI: latência de leitura) devem ser concluídas em menos de 250ms (SLO) nos últimos 28 dias".
    \item \textbf{Orçamento de Erro (Error Budget):} O orçamento de erro é o complemento do SLO (ou seja, $100\% - SLO\%$) \cite{google_sre_book_main}. Ele representa a quantidade de falhas "permitidas" (ex: requisições lentas ou com erro) durante o período. O orçamento de erro é uma ferramenta de gerenciamento: enquanto houver orçamento, a equipe de desenvolvimento tem "permissão" para lançar novas funcionalidades (que inerentemente trazem risco); se o orçamento se esgotar, o foco da equipe deve mudar para a melhoria da confiabilidade \cite{google_sre_book_main}.
\end{itemize}

Esses fundamentos orientam a abordagem metodológica detalhada na Seção \ref{sec:metodologia}, que explica como o planejamento dos SLOs, a infraestrutura como código e os experimentos com k6 e Cloud Tasks foram conduzidos para gerar as evidências analisadas posteriormente.

\section{Metodologia}
\label{sec:metodologia}

O estudo foi conduzido de ponta a ponta, do planejamento dos objetivos de nível de serviço (SLOs) à coleta e interpretação dos experimentos. O enfoque é aplicado e experimental: toda a instrumentação foi construída diretamente no repositório ValorizeAI, o que permite a reprodução dos resultados.

\subsection{Tipo de Pesquisa e Estratégia Geral}

O trabalho caracteriza-se como uma \textbf{pesquisa aplicada} conduzida como \textbf{estudo de caso} de um sistema real em produção. A estratégia seguiu quatro fases iterativas. No \textbf{planejamento}, foram definidos os SLOs (latência P95 de 250~ms, erro $<$0{,}5\%, disponibilidade $\geq$99{,}5\%), mapeadas as cotas vigentes do Cloud Run (10 instâncias de 1~vCPU / 1~GiB, totalizando 10 vCPU) e estimado como essa limitação poderia afetar o throughput — nos ensaios preliminares o workload saturou próximo de 900 RPS, valor usado apenas como referência empírica. Em seguida veio a \textbf{preparação do ambiente}: módulos Terraform provisionaram rede, bancos e serviços gerenciados; Docker Compose reproduziu localmente PostgreSQL, Redis e a stack de observabilidade; o Makefile encapsulou tarefas de lint, testes e execução dos cenários. Na etapa de \textbf{execução controlada}, os cenários k6 de leitura e leitura/escrita foram disparados contra a API em Cloud Run enquanto o pipeline assíncrono recebia um lote adicional de tarefas no Cloud Tasks, exercitando os workers HTTP. Por fim, na \textbf{coleta e análise}, as métricas agregadas (latência, throughput, taxa de erro) foram extraídas dos CSVs e painéis do Cloud Monitoring, e as observações qualitativas sobre o teste de filas foram registradas juntamente com o tempo total de drenagem, subsidiando os capítulos de implementação e resultados.

\subsection{Arquitetura do Ambiente Experimental}

A Figura \ref{fig:arquitetura} sintetiza os componentes usados nos experimentos. O tráfego HTTP/HTTPS entra por um \textbf{Cloud Load Balancer} com \textbf{Cloud CDN}, que reduz a latência de \textit{assets} estáticos e protege o backend com inspeção WAF. Esse tráfego é encaminhado para dois serviços Cloud Run:
\begin{itemize}
    \item \textbf{API Laravel}: processa requisições REST, expõe endpoints usados pelos testes k6 e orquestra o pipeline assíncrono.
    \item \textbf{Laravel Reverb}: mantém conexões WebSocket persistentes para eventos em tempo real; é tratado como serviço independente para permitir escalonamento específico.
\end{itemize}

Ambos os serviços acessam o \textbf{Memorystore for Redis}, usado simultaneamente como cache de leitura (padrão \textit{cache-aside}) e como \textit{backplane} Pub/Sub do Reverb. O armazenamento transacional permanece no \textbf{Cloud SQL for PostgreSQL}, que atende às operações de leitura e escrita executadas durante os testes. Para workloads assíncronos, a API publica tarefas em \textbf{Cloud Tasks}, que aciona workers HTTP também hospedados no Cloud Run. Artefatos grandes (extratos e relatórios) são persistidos no \textbf{Cloud Storage}, mas não fizeram parte dos testes de carga.

\begin{figure}[ht]
    \centering
    \resizebox{\linewidth}{!}{%
    \begin{tikzpicture}[node distance=1.5cm, every node/.style={font=\footnotesize, align=center}]
        \node (cdn) [draw, rounded corners, fill=gray!15, minimum width=5cm, minimum height=0.9cm] {Cloud Load Balancer + Cloud CDN};
        \node (api) [draw, rounded corners, fill=blue!10, minimum width=3cm, minimum height=0.9cm, below left=1.1cm and 2.0cm of cdn] {Cloud Run\\API};
        \node (reverb) [draw, rounded corners, fill=blue!10, minimum width=3cm, minimum height=0.9cm, below=1.1cm of cdn] {Cloud Run\\Reverb};
        \node (workers) [draw, rounded corners, fill=blue!10, minimum width=3cm, minimum height=0.9cm, below right=1.1cm and 2.0cm of cdn] {Cloud Run\\Workers};
        \node (shared) [draw, rounded corners, fill=orange!15, minimum width=6cm, minimum height=1.2cm, below=1.3cm of reverb] {Cloud SQL + Memorystore (Redis) + Cloud Storage};
        \node (tasks) [draw, rounded corners, fill=green!10, minimum width=5cm, minimum height=0.9cm, below=1.0cm of shared] {Cloud Tasks};

        \draw[->, thick] (cdn) -- (api);
        \draw[->, thick] (cdn) -- (reverb);
        \draw[->, thick] (cdn) -- (workers);
        \draw[->, thick] (api) -- (shared);
        \draw[->, thick] (reverb) -- (shared);
        \draw[->, thick] (workers) -- (shared);
        \draw[->, thick] (api) |- (tasks);
        \draw[->, thick] (tasks) -| (workers);
    \end{tikzpicture}}
    \caption{Arquitetura utilizada nos experimentos.}
    \label{fig:arquitetura}
\end{figure}

\subsection{Ferramentas e Processo de Preparação}

Do ponto de vista de engenharia, três pilares garantiram a reprodutibilidade:
\textbf{(i)} \emph{Infraestrutura como Código}: os módulos Terraform descrevem VPC, balanceadores, Cloud Run, Cloud SQL, Redis e Cloud Tasks. Cada mudança passa por \textit{plan/apply} versionado, evitando deriva de ambiente.
\textbf{(ii) Ambientes determinísticos}: o Makefile e os manifestos Docker recompõem o stack local (PostgreSQL, Redis, Loki/Tempo e PHP 8.4) idêntico ao ambiente de teste antes de qualquer execução k6.
\textbf{(iii) Observabilidade}: OpenTelemetry + Cloud Monitoring coletam métricas de latência, uso CPU/memória e backlog de filas, permitindo correlacionar cada rodada com os SLOs definidos.

\subsection{Planejamento dos SLOs e Desenho dos Cenários}

Com base nas premissas de negócio e na literatura de SRE \cite{mccoy_slo_2020,google_sre_book_main}, o sistema foi avaliado contra três metas: latência P95 $\leq 250$~ms, taxa de erro $<$ 0{,}5\% e disponibilidade mensal $\geq 99{,}5\%$. A cota vigente do Cloud Run (10 instâncias de 1~vCPU/1~GiB) limita o total de CPU disponível; no nosso cenário isso significou que os testes deveriam aumentar a carga até consumir essas 10 vCPU (o que, empiricamente, ocorreu perto de 900 RPS), documentando o comportamento imediatamente antes do esgotamento.

Dois cenários foram modelados:
\begin{enumerate}
    \item \textbf{Leitura intensiva}: 1{.}000 usuários virtuais consultando listas de transações por 17 minutos em seis estágios, exercitando cache Redis + réplica de leitura do PostgreSQL.
    \item \textbf{Mistura leitura/escrita}: 650 usuários virtuais alternando consultas e criação de transações durante 21 minutos, forçando locks no banco e pressionando o pipeline de escrita.
\end{enumerate}
Além desses ensaios HTTP, foi planejado um \textbf{teste de filas} no qual um volume elevado de tarefas artificiais percorre o fluxo Cloud Tasks → workers HTTP, permitindo observar o tempo de drenagem e a elasticidade dos consumidores assíncronos.

\subsection{Execução dos Experimentos}

Cada rodada segue os passos:
\begin{enumerate}
    \item \textbf{Preparação dos dados}: seeds e factories povoam o PostgreSQL com contas, transações e orçamentos realistas; a instância Redis é pre-aquecida com métricas e dashboards frequentes.
    \item \textbf{Disparo do cenário}: os perfis do k6 focados em leitura e no mix leitura/escrita são executados via Makefile, apontando para o domínio público do Cloud Load Balancer; estágios, VUs e SLIs monitorados seguem o planejamento experimental.
    \item \textbf{Registro automático}: os resultados agregados são gravados em CSVs (latência, taxa de erro, uso de VUs) e correlacionados com as métricas de infraestrutura capturadas pelo Cloud Monitoring.
    \item \textbf{Teste de filas}: um script HTTP produz um lote adicional de tarefas e a drenagem é acompanhada por meio das métricas do Cloud Tasks e dos logs dos workers.
\end{enumerate}

\subsection{Coleta e Integração das Evidências}

As evidências produzidas sustentam as análises de arquitetura, implementação e resultados:
\begin{itemize}
    \item \textbf{Planilhas de latência e throughput}: derivadas dos CSVs exportados pelo k6, utilizadas posteriormente para comparar as métricas observadas com os SLOs.
    \item \textbf{Series temporais de infraestrutura}: capturas dos dashboards do Cloud Monitoring registram uso de CPU das instâncias Cloud Run, saturação do Redis e backlog do Cloud Tasks durante cada rodada.
    \item \textbf{Relatos de execução}: cada rodada é registrada em um diário experimental com horários, parâmetros e observações qualitativas sobre o comportamento do sistema.
\end{itemize}

Essa metodologia garante rastreabilidade completa entre arquitetura, implementação e resultados, pois cada passo experimental está ancorado em artefatos versionados do projeto.

\section{Implementação}
\label{sec:implementacao}

Esta seção descreve a implementação do ValorizeAI, enfatizando os elementos arquiteturais, os fluxos de execução e os componentes que foram exercitados nos experimentos de carga. O objetivo é apresentar como o sistema opera internamente, de modo a contextualizar os resultados apresentados na próxima seção.

\subsection{Arquitetura Lógica da Aplicação}

A aplicação segue uma arquitetura modular composta por três subsistemas principais:

\begin{itemize}
    \item \textbf{Serviço HTTP (API Laravel):} responsável pelas operações transacionais síncronas (consultas, criação de transações, ingestão de dados e automações).
    \item \textbf{Servidor WebSocket (Laravel Reverb):} dedicado ao envio de notificações em tempo real e atualização dos dashboards.
    \item \textbf{Workers HTTP (Cloud Run):} executores de tarefas assíncronas disparadas pelo Cloud Tasks.
\end{itemize}

Cada subsistema é implantado como serviço independente no Cloud Run, permitindo escalonamento isolado e controle fino de concorrência.

\subsection{Fluxo Síncrono: API HTTP}

O backend Laravel organiza o domínio financeiro em módulos que seguem Clean Architecture e DDD. As rotas públicas acessam controladores que delegam:

\begin{itemize}
    \item \textbf{consultas} para classes \textit{Query}, otimizadas para leitura e usando Redis como cache;
    \item \textbf{escritas} para classes \textit{Action}, que encapsulam validações, regras de negócio, transações no PostgreSQL e emissão de eventos.
\end{itemize}

Os endpoints exercitados nos testes k6 incluem:

\begin{itemize}
    \item \verb|GET /api/transactions|, leitura intensiva de listas paginadas;
    \item \verb|POST /api/transactions|, criação de transações, fluxo que ativa lógica de consistência e atualizações derivadas;
    \item \verb|GET /api/accounts|, consulta leve usada para refletir mudanças de saldo.
\end{itemize}

O caminho de leitura foi otimizado com cache \textit{cache-aside}: a primeira consulta popula Redis, e subsequentes retornam em baixa latência.
O caminho de escrita, por sua vez, não usa cache e realiza operações ACID no PostgreSQL.
Esse fluxo é ilustrado na Figura~\ref{fig:fluxo-sincrono}.

\begin{figure}[htbp]
    \centering
    \begin{tikzpicture}[
        node distance=0.9cm and 1.8cm,
        >=Latex,
        box/.style={
            draw,
            rounded corners,
            align=center,
            minimum width=3.4cm,
            minimum height=0.9cm,
            font=\footnotesize
        }
    ]
        % Linha principal
        \node[box] (client) {Cliente / gerador de carga (k6)};
        \node[box, below=of client] (controller) {Controladores HTTP\\(Laravel)};

        % Split Query / Action
        \node[box, below left=of controller]  (query)  {Camada de leitura\\(\textit{Query})};
        \node[box, below right=of controller] (action) {Camada de escrita\\(\textit{Action})};

        % Armazenamento
        \node[box, below=of query]  (redis) {Redis (cache)};
        \node[box, below=of action] (pg)    {PostgreSQL (ACID)};

        % Setas principais
        \draw[->] (client) -- node[right]{HTTP} (controller);
        \draw[->] (controller) -- node[left]{consultas} (query);
        \draw[->] (controller) -- node[right]{escritas} (action);

        % Leitura: cache-aside
        \draw[->] (query) -- node[left]{consulta / cache-aside} (redis);
        \draw[->] (query) -- node[right]{leitura} (pg);

        % Escrita: só banco
        \draw[->] (action) -- node[right]{transações ACID} (pg);
    \end{tikzpicture}
    \caption{Fluxo síncrono da API HTTP, destacando a separação entre caminhos de leitura (\textit{Query} + Redis) e escrita (\textit{Action} + PostgreSQL).}
    \label{fig:fluxo-sincrono}
\end{figure}


\subsection{Fluxo Assíncrono: Cloud Tasks e Workers}

Tarefas que exigem maior tempo de processamento são delegadas ao pipeline assíncrono. O processo ocorre em quatro etapas:

\begin{enumerate}
    \item a API cria uma tarefa via Cloud Tasks, anexando o payload necessário;
    \item o serviço envia uma requisição HTTP \textit{push} para o endpoint do worker;
    \item o Cloud Run instancia dinamicamente quantos workers forem necessários para consumir o backlog;
    \item cada worker executa o processamento (ex.: importação de extratos, geração de relatórios, triggers de automações).
\end{enumerate}

O uso de \textit{push queues} elimina a necessidade de processos consumidores contínuos e garante elasticidade automática baseada no ritmo de produção.

Esse fluxo é ilustrado na Figura~\ref{fig:fluxo-assincrono}.

\begin{figure}[htbp]
    \centering
    \begin{tikzpicture}[
        node distance=0.9cm,
        >=Latex,
        box/.style={
            draw,
            rounded corners,
            align=center,
            minimum width=3.4cm,
            minimum height=0.9cm,
            font=\footnotesize
        }
    ]
        % Coluna vertical
        \node[box] (client) {Cliente / gerador de carga (k6)};
        \node[box, below=of client] (api) {API HTTP\\(Cloud Run + Laravel)};
        \node[box, below=of api] (tasks) {Cloud Tasks\\(fila HTTP \emph{push})};
        \node[box, below=of tasks] (worker) {Workers HTTP\\(Cloud Run)};
        \node[box, below=of worker] (db) {PostgreSQL / Redis};

        % Setas
        \draw[->] (client) -- node[right]{HTTP} (api);
        \draw[->] (api) -- node[right]{Criação de tarefas} (tasks);
        \draw[->] (tasks) -- node[right]{Entrega \emph{push}} (worker);
        \draw[->] (worker) -- node[right]{Processamento ACID / eventos} (db);
    \end{tikzpicture}
    \caption{Fluxo assíncrono entre API HTTP, Cloud Tasks e workers em Cloud Run.}
    \label{fig:fluxo-assincrono}
\end{figure}

\subsection{Comunicação em Tempo Real: WebSockets com Reverb e Redis}

O servidor WebSocket do Reverb mantém conexões persistentes com os clientes do painel em tempo real. Como o ambiente Cloud Run escala horizontalmente e instancia múltiplos contêineres, cada instância possui um conjunto próprio de clientes conectados.

Para distribuir eventos entre elas, utiliza-se Redis como \textit{backplane} Pub/Sub:

\begin{itemize}
    \item Instâncias publicam eventos em canais Redis.
    \item Todas as instâncias inscritas recebem as mensagens.
    \item A instância que possui o cliente conectado retransmite o evento via WebSocket.
\end{itemize}

Esse mecanismo assegura coerência e funcionamento correto mesmo em ambientes altamente elásticos.

A Figura~\ref{fig:reverb-redis} resume a relação entre instâncias do Reverb, Redis e clientes conectados

\begin{figure}[htbp]
    \centering
    \begin{tikzpicture}[
        node distance=0.9cm and 2.0cm,
        >=Latex,
        box/.style={
            draw,
            rounded corners,
            align=center,
            minimum width=3.4cm,
            minimum height=0.9cm,
            font=\footnotesize
        }
    ]
        % Camada superior: produtores de eventos
        \node[box] (api) {API HTTP / Workers};
        % \node[box, right=of api] (worker) {Workers HTTP};

        % Redis central (abaixo da API, mais ou menos no meio)
        \node[box, below=of api] (redis) {Redis (Pub/Sub)};

        % Reverb logo abaixo do Redis
        \node[box, below left=of redis]  (reverbA) {Reverb -- instância A};
        \node[box, below right=of redis] (reverbB) {Reverb -- instância B};

        % Clientes conectados
        \node[box, below=of reverbA] (client1) {Clientes conectados};
        \node[box, below=of reverbB] (client2) {Clientes conectados};

        % Setas de eventos da aplicação para o Redis
        \draw[->] (api) -- node[left]{Eventos} (redis);
        % \draw[->] (worker) |- node[right]{Eventos} (redis);

        % Pub/Sub entre Redis e Reverb
        \draw[<->] (redis) -- node[left]{Pub/Sub} (reverbA);
        \draw[<->] (redis) -- node[right]{Pub/Sub} (reverbB);

        % WebSockets para os clientes
        \draw[<->] (reverbA) -- node[left]{WebSocket} (client1);
        \draw[<->] (reverbB) -- node[right]{WebSocket} (client2);
    \end{tikzpicture}
    \caption{Distribuição de eventos em tempo real via Redis e instâncias do servidor WebSocket Reverb.}
    \label{fig:reverb-redis}
\end{figure}

\FloatBarrier
\subsection{Modelo de Dados e Otimizações}

O modelo lógico (Figura~\ref{fig:modelo-dados}) segue um desenho multi-inquilino composto por usuários, contas, categorias e orçamentos que representam a estrutura organizacional do sistema, além de transações e divisões (\textit{splits}) responsáveis por manter a granularidade das movimentações financeiras. Embeddings vetoriais, armazenados via \texttt{pgvector}, complementam o modelo ao fornecer suporte para classificação automática e comparação semântica de registros, justificando sua presença mesmo em um ambiente relacional tradicional.

\begin{figure}[H]
    \centering
    \resizebox{0.95\linewidth}{!}{%
    \begin{tikzpicture}[
        table/.style={draw, rounded corners, fill=gray!10, minimum width=4.2cm, minimum height=1.1cm, font=\scriptsize, align=center},
        every edge/.style={->, very thick, >=Stealth}
    ]
        \node (users) at (0,1.5) [table] {Users\\(id, email, preferred\_currency)};
        \node (banks) at (-5.5,-1.5) [table] {Banks\\(id, code, name)};
        \node (accounts) at (0,-1.5) [table] {Accounts\\(user\_id, bank\_id, balance)};
        \node (transactions) at (0,-5.0) [table] {Transactions\\(account\_id, category\_id, amount, date)};
        \node (splits) at (0,-8.5) [table] {Transaction Splits\\(transaction\_id, category\_id, amount)};

        \node (categories) at (6.5,-1.5) [table] {Categories\\(id, user\_id, color)};
        \node (embeddings) at (-6.5,-5.0) [table] {Transaction Embeddings\\(transaction\_id, vector)};
        \node (budgets) at (12.5,1.5) [table] {Budgets\\(user\_id, category\_id, currency)};
        \node (allocations) at (12.5,-1.5) [table] {Budget Allocations\\(budget\_id, month, amount)};

        \draw (users) -- node[midway,left]{1:N} (accounts);
        \draw (banks) -- node[midway,below]{1:N} (accounts);
        \draw (accounts) -- node[midway,left]{1:N} (transactions);
        \draw (transactions) -- node[midway,left]{1:N} (splits);
        \draw (transactions) -- node[midway,above]{1:N} (embeddings);

        \draw (users) -- node[midway,above]{1:N} (budgets);
        \draw (categories) -- node[midway,above]{1:N} (budgets);
        \draw (budgets) -- node[midway,right]{1:N} (allocations);

        \draw (categories) to[out=-140,in=0] node[midway,below]{1:N} (transactions);
        \draw (categories) to[out=-90,in=0] node[midway,right]{1:N} (splits);
    \end{tikzpicture}}
    \caption{Modelo lógico central derivado do esquema relacional do ValorizeAI.}
    \label{fig:modelo-dados}
\end{figure}

Para sustentar o volume de leituras e escritas observado, foram aplicadas otimizações específicas. Índices compostos em \verb|transactions(date, account_id)| aceleram filtros combinados por data e conta, enquanto a padronização da paginação reduz o fan-out em cenários com alto paralelismo. A projeção de DTOs limita a transferência de dados apenas ao necessário para o front-end, preservando largura de banda e diminuindo o trabalho de serialização. Por fim, o pré-aquecimento da cache com consultas frequentes garante que o Redis já contenha os conjuntos de dados mais requisitados antes dos testes, minimizando latência inicial e variações entre execuções.

\subsection{Observabilidade e Instrumentação}

A coleta de evidências envolveu um conjunto integrado de ferramentas. O \textbf{Cloud Monitoring} forneceu métricas de CPU, memória, instâncias ativas, latência e backlog do Cloud Tasks, permitindo enxergar a reação da infraestrutura conforme as cargas variavam. Os \textbf{CSVs automatizados do k6} serviram como fonte canônica para SLIs de latência e taxa de erro, garantindo comparabilidade entre execuções. \textbf{Logs estruturados} foram emitidos para cada requisição crítica a fim de rastrear saturação do banco e falhas de escrita, enquanto \textbf{dashboards personalizados} consolidaram esses dados para acompanhar em tempo real o comportamento dos serviços do Cloud Run durante os experimentos. Essa infraestrutura correlaciona aplicação, banco, Redis, filas e WebSockets com precisão suficiente para justificar as conclusões da Seção~\ref{sec:resultados}.

\subsection{Síntese da Implementação}

A integração entre API síncrona, pipeline assíncrono via Cloud Tasks, cache Redis e servidor WebSocket escalável fornece a base operacional necessária para avaliar os SLOs definidos. O comportamento observado sob tráfego elevado é consequência direta dessas decisões: a API concentra o domínio e delega trabalho pesado aos workers, o Redis mantém leituras rápidas e sincroniza eventos em tempo real, e o Reverb garante que notificações persistam mesmo com escalonamento horizontal. Juntos, esses componentes amarram as otimizações descritas ao longo da seção e sustentam os experimentos discutidos posteriormente.

\section{Resultados e Discussão}
\label{sec:resultados}

Esta seção consolida as métricas obtidas nos testes de carga (k6) e no ensaio assíncrono com Cloud Tasks. As análises seguem os SLIs definidos na metodologia: latência P95, taxa de erro e comportamento das filas sob acúmulo de tarefas. O objetivo é verificar a aderência da arquitetura aos SLOs estabelecidos e compreender os pontos de saturação observados.

\begin{table}[ht]
    \centering
    \caption{Resumo dos cenários de carga e conformidade com os SLOs}
    \label{tab:resultados_k6}
    \begin{tabular}{lcccc}
        \toprule
        Cenário & Latência P95 & Throughput médio & Throughput pico & Taxa de erro \\
        \midrule
        Leitura intensiva & 158~ms & 470 req/s & 970 req/s & 0{,}00\% \\
        Mistura leitura/escrita & 658~ms & 226 req/s & 450 req/s & 0{,}00\% \\
        \bottomrule
    \end{tabular}
\end{table}

Os valores de \textit{throughput} foram extraídos diretamente dos dashboards do Cloud Run, refletindo o número real de requisições por segundo atendidas por todas as instâncias durante cada estágio do teste. Assim, VUs e req/s representam perspectivas complementares do mesmo nível de pressão sobre o sistema.

\subsection{Cenário de Leitura Intensiva}

O cenário de leitura mobilizou 1.000 usuários virtuais por 17 minutos, sustentando em média 470 requisições por segundo e alcançando um pico próximo a 970 req/s no trecho final ($900\rightarrow 1.000$ VUs). A latência P95 permaneceu em 158~ms (Tabela~\ref{tab:resultados_k6}), substancialmente abaixo do SLO de 300~ms, e nenhuma requisição apresentou erro. Esses resultados confirmam que o caminho otimizado de leitura — CDN, API em Cloud Run, Redis como cache e consultas eficientes no PostgreSQL — suporta picos significativos de tráfego sem degradação perceptível.

O Cloud Monitoring registrou o escalonamento completo das 10 instâncias de Cloud Run, atingindo CPU média de 72\% e uso de memória em torno de 31\%. Ou seja, a cota máxima de CPU foi totalmente utilizada, mas sem sinais de exaustão ou sobrecarga que pudessem comprometer as latências observadas.

\begin{figure}[ht]
    \centering
    \includegraphics[width=0.95\linewidth]{figures/fig-read-pxx.png}
    \caption{Latências P50/P95/P99 por carga (VUs) no cenário de leitura. A linha tracejada indica o SLO de 300~ms, não violado durante o teste.}
    \label{fig:latencia-leitura}
\end{figure}

\subsection{Cenário Misto (Leitura/Escrita)}

O cenário misto utilizou provisionamento automático de contas e tokens por VU, eliminando o gargalo artificial presente em versões anteriores do teste.
Com 650 usuários alternando entre 65\% de leituras e 35\% de escritas durante 21 minutos, o sistema sustentou 226 requisições por segundo ($\approx 283$ mil iterações) sem erros e manteve o SLO de 300~ms até aproximadamente 450 RPS.
A violação sistemática do SLO ocorre por volta de 539 VUs, ponto a partir do qual o Cloud Run volta a atingir o limite de 10 instâncias.

Nos estágios acima de 550 VUs, o P95 consolidado do ensaio alcançou 658~ms, enquanto o p99 atingiu 2,67~s. A origem da degradação é clara: o caminho de escrita demanda mais CPU por requisição (validações, transações ACID e invalidação de cache), reduzindo a capacidade de requisições por instância quando o limite de contêineres é atingido. As métricas do Cloud SQL mostraram estabilidade, indicando que o banco relacional não foi o gargalo primário nesse experimento.

\begin{figure}[ht]
    \centering
    \includegraphics[width=0.95\linewidth]{figures/fig-mixed-pxx.png}
    \caption{Latências P50/P95/P99 por carga (VUs) no cenário misto. A faixa tracejada marca o SLO de 300~ms; etapas acima de 539~VUs apresentam violações persistentes.}
    \label{fig:latencia-mix}
\end{figure}

A Figura~\ref{fig:comparacao} ilustra o distanciamento entre os comportamentos das duas cargas. Enquanto o cenário de leitura mantém o P95 abaixo de 200~ms mesmo no pico de 1.000 VUs, o cenário misto diverge abruptamente após 500 VUs. Esse comportamento é consistente com modelos teóricos de sistemas transacionais \cite{kleppmann_ddia_2017}, segundo os quais rotas de escrita têm custo marginal crescente e menor paralelização efetiva, sobretudo em sistemas baseados em contêineres com cotas rígidas de CPU.

\begin{figure}[ht]
    \centering
    \includegraphics[width=0.95\linewidth]{figures/fig-compare-read-vs-mixed.png}
    \caption{Comparação entre os P95 dos cenários de leitura e misto. A leitura permanece estável; a escrita viola o SLO acima de 539~VUs.}
    \label{fig:comparacao}
\end{figure}

Os resultados reforçam que, antes de adotar mecanismos mais complexos (particionamento, réplicas dedicadas ou CQRS físico), a arquitetura pode obter ganhos expressivos liberando o limite de instâncias HTTP no Cloud Run ou reduzindo o custo computacional por escrita.

\subsection{Processamento Assíncrono com Cloud Tasks}

O ensaio assíncrono publicou 51{,}58 mil tarefas em lote e monitorou a drenagem entre 01:08 e 01:18 ($\approx 10$ minutos). Isso corresponde a aproximadamente 86 tarefas/segundo processadas em média, com variação alinhada ao número de instâncias de worker ativadas dinamicamente. O Cloud Run escalou horizontalmente enquanto havia backlog, reduzindo o número de instâncias assim que o volume de tarefas diminuiu.

Nenhuma entrega foi perdida ou duplicada. A latência ponta-a-ponta permaneceu estável porque:

\begin{itemize}
    \item o Cloud Tasks opera em modo \textit{push}, eliminando \textit{polling};
    \item o Redis atuou somente como \textit{backplane} de eventos e não participou do pipeline crítico;
    \item cada worker pôde operar de forma idempotente e independente.
\end{itemize}

Esse comportamento demonstra elasticidade eficiente: a infraestrutura permanece mínima em períodos ociosos e escala agressivamente apenas durante janelas de pico, alinhando custo e demanda sem intervenção manual.

\begin{figure}[ht]
    \centering
    \includegraphics[width=0.95\linewidth]{figures/fig-queue-drain.png}
    \caption{Taxa de processamento das filas (tarefas/min) e janela de drenagem entre 01:08 e 01:18 no ensaio com Cloud Tasks.}
    \label{fig:filas}
\end{figure}

\subsection{Síntese e Discussão Integrada}

Os resultados demonstram que:

\begin{itemize}
    \item \textbf{O caminho de leitura é altamente escalável}, atendendo 1.000 VUs com folga e sem violar SLOs.
    \item \textbf{O caminho de escrita é limitado pela cota de CPU do Cloud Run}, não pelo banco.
    \item \textbf{O pipeline assíncrono é elástico e confiável}, drenando grandes volumes sem perda de tarefas.
    \item \textbf{A arquitetura é adequada para workloads dominados por leitura}, mas exige ajustes para workloads de escrita concorrente.
\end{itemize}

Esses achados orientam recomendações para escalabilidade futura, incluindo aumento das cotas de instâncias, separação de serviços de escrita, redução do custo computacional das rotas críticas e, eventualmente, adoção de padrões arquiteturais como CQRS físico ou particionamento de dados.

\subsection{Análise de Custos Operacionais e Trade-offs de Elasticidade}
\label{sec:analise_custos}

Embora o foco principal deste trabalho esteja no desempenho e na validação dos SLOs, a análise de custos é essencial para avaliar a viabilidade da arquitetura. Os serviços utilizados — especialmente o Cloud Run — possuem modelos de cobrança distintos dos ambientes tradicionais baseados em máquinas virtuais (Compute Engine), o que altera significativamente a dinâmica econômica do sistema.

\subsubsection*{Custo por vCPU-second: Cloud Run vs Compute Engine}

Nos preços vigentes para a região \textit{southamerica-east1} (São Paulo), o Cloud Run apresenta o seguinte custo:

\begin{itemize}
    \item \textbf{Cloud Run — CPU ativa}: \$0.0000336 por vCPU-second
    \item \textbf{Cloud Run — CPU ociosa (instância mínima)}: \$0.0000035 por vCPU-second
\end{itemize}

Já o Compute Engine, para os mesmos descontos flexíveis, possui custo efetivamente menor por vCPU-second:

\begin{itemize}
    \item \textbf{Compute Engine (via Compute Flex CUD 3 anos)}: \$0.000011664 por vCPU-second
\end{itemize}

Essa diferença implica que, em termos unitários, \textbf{o Cloud Run cobra entre 2,8× e 3× mais por vCPU-second ativo do que uma VM equivalente}. Na prática:

\[
\frac{0.0000336}{0.000011664} \approx 2.88
\]

Em um cenário puramente estático, o Compute Engine seria mais barato para workloads intensivos.

Entretanto, essa comparação desconsidera o fator determinante: \textbf{elasticidade}. Enquanto o Compute Engine mantém vCPUs provisionadas 24/7, independentemente da carga, o Cloud Run cobra CPU e memória apenas durante o processamento de requisições — reduzindo automaticamente as instâncias a zero em períodos ociosos.

Assim, mesmo tendo custo unitário maior, o Cloud Run elimina por completo o \textit{idle compute}, característica crítica em sistemas com tráfego variável, como o ValorizeAI.

\subsubsection*{Impacto da ociosidade em workloads transacionais}

Para suportar picos como os medidos no experimento de leitura (pico de $\approx$970 req/s), uma VM precisaria:

\begin{itemize}
    \item múltiplos vCPUs constantemente ativos;
    \item memória proporcional ao número de workers simultâneos;
    \item dimensionamento estático capaz de absorver o pior caso.
\end{itemize}

O custo dessa abordagem é permanente: a VM permanece ativa e cobrada integralmente mesmo durante horas de baixa demanda, gerando desperdício estrutural.

No Cloud Run, as 10 instâncias só foram ativadas durante os picos de carga — exatamente quando necessário. Fora dessas janelas, o custo cai próximo de zero. Assim, o Cloud Run converte \textit{picos irregulares em custo proporcional}.

\subsubsection*{GKE como alternativa intermediária}

O Google Kubernetes Engine (GKE) permite um equilíbrio entre custo e flexibilidade: usa nós Compute Engine (com preços menores por vCPU) e incorpora escalonamento horizontal por meio de HPA/VPA.

Contudo, isso introduz dois desafios:

\begin{enumerate}
    \item \textbf{Complexidade operacional}: gestão de nodes, upgrades, autoscaler, limites, requests, probes, políticas de interrupção, etc.
    \item \textbf{Menor eficiência de escalonamento}: o HPA/VPA reage mais lentamente a picos súbitos, podendo gerar:
    \begin{itemize}
        \item janelas com CPU ociosa (custo desperdiçado);
        \item atrasos no escalonamento (piorando latências e violando SLOs).
    \end{itemize}
\end{enumerate}

Assim, embora o custo por vCPU do GKE possa se aproximar ao do Compute Engine, a probabilidade de manter recursos ociosos aumenta, reduzindo a vantagem financeira.

\subsubsection*{Resumo Comparativo}

\begin{table}[ht]
\centering
\caption{Comparação econômica entre Cloud Run, GKE e Compute Engine (southamerica-east1).}
\label{tab:comparacao_custos}
\begin{tabular}{lccc}
\toprule
\textbf{Serviço} & \textbf{Custo vCPU/s} & \textbf{Elasticidade} & \textbf{Ociosidade esperada} \\
\midrule
Cloud Run (ativo) & \$0.0000336 & Automática, granular & Muito baixa \\
Compute Engine (VM) & \$0.000011664 & Nenhuma & Alta \\
GKE (nodes CE) & \$0.000011664 & Média (HPA/VPA) & Moderada \\
Cloud Run (mínimo ocioso) & \$0.0000035 & Automática & Baixa \\
\bottomrule
\end{tabular}
\end{table}

\subsubsection*{Conclusão econômica}

Apesar do custo por vCPU-second do Cloud Run ser superior ao do Compute Engine, sua \textbf{eliminação total de ociosidade} e sua \textbf{elasticidade altamente responsiva} fazem com que o custo real seja substancialmente menor para workloads com tráfego irregular, como os exercitados no ValorizeAI. GKE permite custos intermediários, porém à custa de maior complexidade e risco de ineficiência no autoscaling.

Em outras palavras: \textbf{o custo unitário mais alto do Cloud Run é compensado pela eficiência operacional quando a carga é variável}. Já em cargas previsíveis e constantes, Compute Engine ou GKE tendem a ser mais econômicos.

\begin{figure}[ht]
\centering
\begin{tikzpicture}
\begin{axis}[
    width=0.95\linewidth,
    height=7cm,
    xlabel={Tempo (horas do dia)},
    ylabel={Custo relativo (\$)},
    xmin=0, xmax=24,
    ymin=0, ymax=1.1,
    xtick={0,4,8,12,16,20,24},
    ytick={0,0.25,0.5,0.75,1.0},
    grid=major,
    legend style={at={(0.5,-0.20)},anchor=north,legend columns=3},
    thick,
]

% Compute Engine (custo estável)
\addplot[red, thick] coordinates {
    (0,0.75) (4,0.75) (8,0.75) (12,0.75) (16,0.75) (20,0.75) (24,0.75)
};
\addlegendentry{Compute Engine (custo fixo)}

% Cloud Run (custo proporcional)
\addplot[blue, thick] coordinates {
    (0,0.05) (4,0.05) (8,0.3) (12,1.0) (16,0.4) (20,0.1) (24,0.05)
};
\addlegendentry{Cloud Run (elástico)}

% GKE (curva intermediária)
\addplot[green!60!black, thick] coordinates {
    (0,0.3) (4,0.3) (8,0.45) (12,0.65) (16,0.45) (20,0.35) (24,0.3)
};
\addlegendentry{GKE (elasticidade moderada)}

\end{axis}
\end{tikzpicture}
\caption{Comparação conceitual do custo relativo ao longo do dia entre Compute Engine, Cloud Run e GKE, assumindo tráfego variável com pico ao meio-dia.}
\label{fig:custo-conceitual}
\end{figure}

A Figura~\ref{fig:custo-conceitual} ilustra conceitualmente o comportamento de custos ao longo do dia para três modelos de execução. Em workloads com picos concentrados — como os observados nos cenários deste TCC — o Cloud Run segue o perfil de tráfego, cobrando mais apenas próximo ao pico e retornando praticamente a zero nas horas ociosas. Já o Compute Engine mantém custo constante, independentemente da demanda. O GKE apresenta comportamento intermediário: embora reduza custos fora do pico, o autoscaler tende a manter nós parcialmente ociosos, especialmente em cargas irregulares ou altamente dinâmicas.

\section{Conclusão}
\label{sec:conclusao}

O ValorizeAI demonstrou que é possível documentar e validar, de ponta a ponta, uma arquitetura orientada a eventos construída integralmente sobre serviços gerenciados. O trabalho conectou decisões de design , Cloud Run, Redis, Cloud SQL, Cloud Tasks e Reverb , ao domínio transacional modelado no banco relacional, oferecendo um roteiro explícito para equipes que necessitam de elasticidade sem abrir mão de consistência forte. A metodologia experimental, fundamentada em princípios de SRE, complementou essa documentação ao mostrar como planejar SLOs, provisionar infraestrutura como código e conduzir testes reprodutíveis, reforçando o caráter aplicado da pesquisa.

Os cenários de carga confirmaram a eficácia das otimizações de leitura: com 1.000 usuários virtuais ($\approx 470$ req/s em média e pico de $\approx 970$ req/s), o sistema manteve latência P95 de 158~ms e taxa de erro nula, atendendo plenamente aos SLOs estabelecidos. Já o cenário misto revelou o limiar atual da solução: o patamar de 650 VUs ($\approx 226$ req/s) elevou o P95 para 658~ms e o p99 para 2,67~s, resultado diretamente relacionado ao limite de 10 instâncias de Cloud Run e ao maior custo computacional das rotas de escrita (validações, transações ACID e invalidação de cache). O Cloud SQL manteve tempos estáveis ao longo do teste, indicando que o gargalo predominante permanece na camada HTTP. No plano assíncrono, a drenagem de 51{,}58~mil tarefas em 10~min comprovou que a combinação Cloud Tasks + workers em Cloud Run sustenta rajadas intensas sem intervenção manual, preservando elasticidade e rastreabilidade do processamento.

O percurso deste trabalho seguiu uma narrativa coesa para responder à sua questão central. Partiu-se do \textbf{problema} da escassez de validações integradas de arquiteturas elásticas (a \textbf{lacuna} na literatura), propondo como \textbf{contribuição} um estudo de caso completo e reprodutível. Para isso, os objetivos específicos foram alcançados sequencialmente: a arquitetura foi mapeada e seus fluxos documentados; os testes de carga e de pipeline assíncrono foram executados para validar os SLOs, revelando tanto a robustez do caminho de leitura quanto os limites do de escrita; e, por fim, os resultados foram interpretados para fornecer uma análise de causa e efeito e de custo-benefício. Essa jornada não apenas validou o ValorizeAI, mas entregou um roteiro prático para a evolução de sistemas transacionais modernos.

\subsection{Limitações}

Duas limitações principais moldaram os resultados. Primeiro, a infraestrutura operou com as cotas padrão de um ambiente recém-provisionado (10 instâncias de 1~vCPU), o que restringiu a observação de comportamentos em escalas superiores. Segundo, o banco de dados permaneceu centralizado em uma única instância regional do Cloud SQL; embora suficiente para o tráfego exercitado, essa configuração pode induzir contenção em cenários de escrita simultânea em grande escala.

\subsection{Trabalhos Futuros}

Como continuidade, pretende-se ampliar progressivamente a cota de Cloud Run, repetindo os testes sob 20 ou 40~vCPU para identificar novos gargalos. Também se planeja avaliar alternativas para aliviar operações analíticas, como réplicas de leitura e ajustes verticais das instâncias primárias para absorver picos de escrita. Outro eixo de evolução envolve mecanismos de \textit{failover} multi-região para banco e Redis, aumentando a resiliência global da plataforma. No plano funcional, o pipeline assíncrono poderá ser estendido com orquestração baseada em eventos (por exemplo, Workflows), além da expansão do uso de embeddings e automações de categorização para casos como detecção de anomalias e recomendações transacionais.

\newpage

% \nocite{BCB_Economia_Bancaria_2024,Febraban_Tecnologia_2024,Febraban_Transacoes_2025}

\bibliographystyle{article-config/sbc}
\bibliography{references/references}

\end{document}
