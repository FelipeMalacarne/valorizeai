\section{Fundamentação Teórica}
\label{sec:fundamentacao}

Esta seção apresenta os conceitos que sustentam o design, a implementação e a validação empírica do ValorizeAI. São abordados princípios de design de software, fundamentos arquiteturais dos componentes utilizados e noções essenciais de Engenharia de Confiabilidade (SRE), que orientam a formulação dos SLOs e os experimentos descritos na metodologia.

\subsection{Princípios de Design de Software}
\label{sec:principios_design}

O desenvolvimento do ValorizeAI segue práticas consolidadas de engenharia de software, que visam reduzir acoplamento, aumentar coesão e facilitar evolução incremental da aplicação. A \textit{Clean Architecture}, proposta por Martin \cite{martin_clean_2017}, sustenta essa base ao impor a \textit{Regra da Dependência}: detalhes variáveis, como frameworks e tecnologias de persistência, devem depender das regras de negócio, e não o contrário. No ValorizeAI, essa diretriz se manifesta na separação explícita entre lógica de domínio (Actions e Queries) e elementos de infraestrutura (controladores, Reverb, Redis e Cloud SQL), preparando o terreno para os demais princípios.

Sobre essa fundação, o \textit{Domain-Driven Design} (DDD) de Evans \cite{evans_ddd_2003} fornece a semântica necessária para lidar com o domínio financeiro. Linguagem ubíqua, contextos delimitados e agregados são utilizados para evitar ambiguidade e garantir que transações, contas e categorias sejam modeladas de forma consistente. A delimitação clara dos contextos orienta a implementação das Actions e Queries, garantindo que cada componente conheça apenas o que precisa conhecer.

Por fim, o sistema otimiza caminhos de leitura e escrita com \textit{Command Query Responsibility Segregation} (CQRS) \cite{fowler_cqrs_2011} e \textit{Data Transfer Objects} (DTOs) \cite{fowler_peaa_2002}. Consultas são tratadas por \textit{Queries} especializadas, que exploram Redis para acesso rápido, enquanto escritas passam por Actions que encapsulam regras de negócio, validações e persistência. DTOs isolam os dados expostos pela API dos modelos internos, reduzindo acoplamento e dando suporte às camadas anteriormente descritas, encerrando o ciclo de princípios que conduzem o design.

\subsection{Arquitetura e Componentes da Aplicação}
\label{sec:componentes_app}

A arquitetura do ValorizeAI combina serviços gerenciados que oferecem elasticidade, isolamento, baixo acoplamento e capacidade de processamento assíncrono, começando pelo Google Cloud Run, plataforma CaaS que executa contêineres \textit{stateless} com escalonamento automático e \textit{scale-to-zero} \cite{google_elasticity_2024}. Nele operam duas classes de contêineres: o serviço HTTP principal e o servidor WebSocket, ambos abastecidos pelos mesmos pacotes de aplicação, porém com perfis de escalonamento distintos. O Cloud Tasks complementa esse cenário ao implementar o componente assíncrono de EDA, enfileirando tarefas que exigem maior tempo de processamento e permitindo que a API principal permaneça responsiva mesmo durante operações intensivas ao delegar trabalho para \textit{workers} dedicados.

Para manter comunicações síncronas consistentes, o Laravel Reverb \cite{laravel_reverb_docs_2025} fornece o canal WebSocket de alta performance apoiado no protocolo Pusher. Seu recurso central é a capacidade de operar em múltiplas instâncias: para garantir que eventos emitidos por qualquer contêiner sejam entregues aos clientes conectados em outro, o Reverb utiliza Redis como \textit{backplane}. Esse Redis, por sua vez, é um armazenamento em memória de baixa latência \cite{kleppmann_ddia_2017} que desempenha dois papéis complementares — acelera consultas críticas como cache de leitura, reduzindo carga sobre o Cloud SQL, e atua como barramento Pub/Sub para sincronizar múltiplas instâncias do Reverb. A interligação entre Cloud Run, Cloud Tasks, Reverb e Redis garante que as metas de elasticidade e responsividade descritas na seção anterior possam ser cumpridas.

\subsection{Engenharia de Confiabilidade de Sites (SRE)}
\label{sec:sre}

A metodologia de validação do ValorizeAI segue os princípios de SRE apresentados pelo Google \cite{google_sre_book_main}, tratando confiabilidade como disciplina quantitativa guiada por indicadores e metas formalizadas. Três conceitos embasam os experimentos: os \textbf{SLIs} (\textit{Service Level Indicators}) capturam métricas como latência P95, taxa de erro ou \textit{throughput} \cite{mccoy_slo_2020}; os \textbf{SLOs} estabelecem metas para esses indicadores (por exemplo, ``95\% das leituras com latência < 300 ms''); e o \textbf{orçamento de erro} define a tolerância máxima para falhas dentro do período do SLO, indicando quando priorizar evolução ou estabilidade. Esses conceitos orientam os testes de carga apresentados na Seção~\ref{sec:metodologia}, guiando a análise de saturação, violações de SLO e comportamento do sistema sob tráfego intenso, e completam a ligação entre princípios de design, arquitetura e validação empírica.
