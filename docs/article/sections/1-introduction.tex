\section{Introdução}
\label{sec:introducao}

O ValorizeAI, objeto deste estudo, surge como um caso completo para investigar a relação entre elasticidade e observabilidade. Trata-se de uma aplicação web modular que integra ingestão de dados, processamento síncrono e assíncrono e comunicação em tempo real, utilizando uma \textit{stack} moderna baseada em Cloud Run, Cloud SQL, Redis (Memorystore), Cloud Tasks, WebSockets e serviços auxiliares.

Aplicações digitais modernas — incluindo plataformas de e-commerce, serviços financeiros e sistemas colaborativos — enfrentam o desafio de absorver cargas de trabalho voláteis sem comprometer a experiência do usuário \cite{google_elasticity_2024}. A resposta predominante tem sido a \textit{elasticidade na nuvem}, isto é, a capacidade de alocar e desalocar recursos automaticamente conforme a demanda varia, substituindo o provisionamento manual por escalonamento em tempo real para evitar desperdícios e manter a responsividade.

Essa elasticidade, implementada com microsserviços, contêineres e paradigmas \textit{serverless}, introduz complexidade que só pode ser controlada com observabilidade avançada \cite{newrelic_observability_2023}. Logs, métricas e \textit{traces} tornam-se insumos para SLIs que, por sua vez, validam e ajustam políticas automáticas de escalonamento. Elasticidade e observabilidade formam, assim, um ciclo de \textit{feedback} que opera em horizontes temporais inviáveis para processos manuais, reforçando a necessidade de arquiteturas e práticas integradas.

\subsection{Justificativa e Problema de Pesquisa}
\label{sec:justificativa}

Workloads transacionais que envolvem ingestão intensa de dados, estados compartilhados e interfaces colaborativas, como os simulados pelo ValorizeAI, impõem requisitos rigorosos: consistência forte, rastreabilidade para auditoria e respostas de baixa latência mesmo sob variações abruptas de tráfego. Para atender a esses requisitos, arquiteturas modernas combinam componentes especializados, como CDNs e balanceadores globais, filas assíncronas orientadas a eventos, cache distribuído e comunicação persistente via WebSockets \cite{barri_scalability_2025, confluent_eda_2024, yadav_redis_leaderboard_2019, fernando_websocket_2025}.

Embora existam estudos pontuais sobre esses componentes, a literatura apresenta lacunas quanto à validação integrada de arquiteturas híbridas (CaaS + filas + WebSockets) em cenários reprodutíveis de carga \cite{wjaets_serverless_ml_2022, abad_serverless_gap_2021}. Trabalhos existentes tendem a focar em comparações de ferramentas de IaC \cite{pessa_iac_2023} ou no desempenho de microsserviços isolados \cite{hebbar_reactive_2025}, raramente considerando o comportamento do sistema completo.

Este trabalho busca preencher essa lacuna ao documentar a arquitetura do ValorizeAI e validar seu comportamento sob estresse de carga frente a SLOs definidos. A questão central investigada é: \textit{Como uma arquitetura híbrida e elástica, composta por Cloud Run, Redis, Cloud SQL, Cloud Tasks e WebSockets dedicados, se comporta sob condições intensas de carga, e como esse comportamento pode ser validado de forma reprodutível?}

\subsection{Objetivo Geral}
\label{sec:objetivo_geral}

Demonstrar, por meio de documentação técnica e experimentos de desempenho, que a arquitetura do ValorizeAI, composta por CDN, contêineres escalados horizontalmente, processamento assíncrono em filas, servidor de WebSockets, armazenamento de artefatos em \textit{buckets} e cache distribuído em Redis, sustenta os SLOs definidos para um produto transacional completo, mantendo código, infraestrutura e observabilidade versionados em repositório.

\subsection{Objetivos Específicos}
\label{sec:objetivos_especificos}

\begin{enumerate}
    \item Mapear a arquitetura \textit{end-to-end}, destacando o papel do balanceador/CDN, instâncias de contêineres, servidor WebSockets, filas assíncronas, \textit{buckets} e Redis.
    \item Documentar o desenvolvimento dos módulos críticos do sistema, incluindo ingestão de dados, automações, notificações e painéis em tempo real.
    \item Planejar e executar testes de carga (k6, cenários de leitura e leitura/escrita) e testes assíncronos, validando a arquitetura frente aos SLOs definidos.
    \item Interpretar os resultados e propor otimizações relacionadas a desempenho, elasticidade e custo.
\end{enumerate}

\subsection{Contribuições Tangíveis}
\label{sec:contribuicoes}

\begin{enumerate}
    \item Arquitetura documentada e replicável.
    \item Infraestrutura reprodutível (Terraform, Docker, Makefile).
    \item Cenários de desempenho registrados e transparentes (k6).
    \item Validação do pipeline assíncrono baseado em Cloud Tasks.
\end{enumerate}

Em conjunto, essas contribuições formam um pacote completo de replicação, código, automações e experimentos, para avaliações futuras de workloads transacionais em ambientes CaaS.

\vspace{0.5em}
O restante deste artigo está organizado da seguinte forma:
a Seção~\ref{sec:relacionados} apresenta os trabalhos relacionados;
a Seção~\ref{sec:fundamentacao} discute a fundamentação teórica;
a Seção~\ref{sec:metodologia} descreve a metodologia experimental;
a Seção~\ref{sec:implementacao} detalha a implementação do ValorizeAI;
a Seção~\ref{sec:resultados} consolida os resultados e análises;
e, por fim, a Seção~\ref{sec:conclusao} apresenta as conclusões e trabalhos futuros.
